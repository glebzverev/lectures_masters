\section{Лекция 3}

\begin{equation*}
	\left.\begin{aligned}
		&Au=f \\
		&u, f \in H
	\end{aligned}\quad \right | \quad \Omega \subset \mathbb{R}^m, \quad H = L_2(\Omega)
\end{equation*}

\[
\begin{cases}
	-\Delta u = f, \qquad f \in C(\overline{\Omega}) \\
	u |_{s} = 0
\end{cases}
\]
\[ D_A = \{u \in C^2 (\overline{\Omega}); \enspace u|_s = 0\} \]
\[ A = -\Delta u \] \\

\textbf{Формула Остроградского}

\[ \int\limits_{\Omega}\left(\frac{\partial \varphi}{\partial x } + \frac{\partial \psi }{\partial y} + \frac{\partial \omega}{\partial y}\right) d\Omega = \int\limits_{S } \biggl( \varphi \cos(\overline{n} \cdot x) + \psi \cos(\overline{n}\cdot y) + \omega \cos (\overline{n} \cdot z) \biggr) dS\]
\[ W =
\begin{pmatrix}
	\varphi \\
	\psi \\
	\omega
\end{pmatrix}
\qquad \qquad \int\limits_{\Omega}^{} \text{div} W d\Omega = \int\limits_{S}^{} W_n dS
\] \\


Пусть $ \varphi = uv, \ \psi = \omega = 0 $

\[ \int\limits_{\Omega} u \frac{\partial v }{\partial x } d \Omega = - \int\limits_{\Omega}^{} v \frac{\partial u }{\partial x } d \Omega + \int\limits_{S}^{} uv \cos(\overline{n} \cdot x) dS \]

\begin{equation}
	\label{3.0}
	\tag{3.0}
	\int\limits_{\Omega } u \frac{\partial v }{\partial x_i} d \Omega = - \int\limits_{\Omega }^{} v \frac{\partial u}{\partial x_i} d \Omega + \int\limits_{S }^{} uv cos(\overline{n} \cdot x_i ) dS \qquad \text{в} \ \mathbb{R}^m
\end{equation} \\

\subsection{Формулы Грина}

\[ Lu = - \sum_{i,k =1}^{m } \frac{\partial }{\partial x_i} \left(A_{ik} (P) \frac{ \partial u(P)}{\partial x_k}\right) + C(P) u(P) \]

\[ D_L = \{ u \in C^2(\overline{\Omega}) \}, \quad P \in \Omega \subset \mathbb{R}^m, \quad C(P) \in C(\overline{\Omega}) \]
\[ A_{ik}(P) \in C(\overline{\Omega}), \quad A_{ik}(P) = A_{ki}(P) \enspace \forall P, \quad i,k=\overline{1,n} \]
\[ \int\limits_{\Omega} v Lu d \Omega = - \sum_{i,k =1}^{m }  \int\limits_{\Omega} v \frac{\partial }{\partial x_i} \left(A_{ik} \frac{ \partial u}{\partial x_k}\right) d \Omega + \int\limits_{\Omega}^{}C u v d \Omega  \]

в \eqref{3.0} подставим $ u \rightarrow v, v \rightarrow A_{ik} \frac{\partial u }{\partial x_k }$

\begin{equation}
	\label{3.1}
	\tag{3.1}
	\int\limits_{\Omega}^{} v L u d\Omega = \int\limits_{\Omega}^{} \sum_{i,k = 1}^{m}A_{ik} \frac{\partial u }{\partial x_k} \frac{\partial v }{\partial x_i} d\Omega + \int\limits_{\Omega} C u vd\Omega - \int\limits_{S }^{} v \sum_{i,k=1}^{m} A_{ik} \frac{\partial u}{\partial x_k } cos(\overline{n} \cdot x_i ) dS
\end{equation}

\begin{equation}
	\label{3.2}
	\tag{3.2}
	\int\limits_{\Omega}^{} u L u d\Omega = \int\limits_{\Omega} \left[ \sum_{i,k = 1}^{m}A_{ik} \frac{\partial u }{\partial x_i} \frac{\partial u }{\partial x_k} + Cu^2 \right] d\Omega - \int\limits_{S } u \sum_{i,k =1}^{m} A_{ik} \frac{\partial u}{\partial x_k } cos(\overline{n} \cdot x_i ) dS
\end{equation}

из \eqref{3.1} вычитаем ее же, но поменяв местами $u$ и $v$: \qquad $ \eqref{3.1} - \eqref{3.1}_{u \rightleftarrows v} $

\begin{equation*}
	\begin{split}
		\int\limits_{\Omega }^{} (vLu - uLv) d\Omega = & \int\limits_{\Omega} \cancelto{0}{\left[ \sum_{i,k = 1}^{m}A_{ik} \frac{\partial u }{\partial x_k} \frac{\partial v }{\partial x_i} - \sum_{i,k = 1}^{m}A_{ik} \frac{\partial u }{\partial x_i} \frac{\partial v }{\partial x_k} \right]} d\Omega \ - \\
		& - \int\limits_{S} \left[ v \sum_{i,k =1}^{m} A_{ik} \frac{\partial u}{\partial x_k } cos(\overline{n} \cdot x_i ) - u \sum_{i,k =1}^{m} A_{ik} \frac{\partial v}{\partial x_k } cos(\overline{n} \cdot x_k )\right] dS
	\end{split}
\end{equation*}

\[ N u := \sum_{i,k=1}^{m } A_{ik } \frac{\partial u }{\partial x_i } cos(\overline{n} \cdot x_i) \]

\begin{equation}
	\label{3.3}
	\tag{3.3}
	\int\limits_{\Omega} \left( vLu - uLv \right) d\Omega = \int\limits_{S} \left( uNv - vNu \right) dS
\end{equation}

Частный случай формул Грина, это оператор Лапласа:

\[ Lu = - \Delta u; \ A_{ii} = 1; \ A_{ik} = 0, \ i \neq k; \ C = 0 \]

\begin{equation}
	\label{3.4}
	\tag{3.4}
	- \int\limits_{\Omega} v \Delta u d\Omega  = \int\limits_{\Omega}^{} \sum_{i=1}^{m } \frac{\partial u }{\partial x_i }  \frac{\partial v }{\partial x_i} d \Omega - \int\limits_{S }^{} v \frac{\partial u }{\partial n } dS
\end{equation}

\begin{equation}
	\label{3.5}
	\tag{3.5}
	- \int\limits_{\Omega} u \Delta u d \Omega = \int\limits_{\Omega}^{} {\left(\frac{\partial u }{\partial x_i}\right)}^2 d \Omega - \int\limits_{S }^{} u \frac{\partial u }{\partial n } dS
\end{equation}

\begin{equation}
	\label{3.6}
	\tag{3.6}
	- \int\limits_{\Omega} (v \Delta u - u \Delta v ) d\Omega = \int\limits_{S }^{} \left(v \frac{\partial u }{\partial n } - u \frac{\partial v}{\partial n}\right) dS
\end{equation}

\subsection{Положительные операторы}

Пусть оператор $A$ симметричен в $H$

\textbf{Опр.} Оператор называется положительным, если $ \forall u \in D_A \subset H, \qquad (Au, u) \geq 0 \Leftrightarrow u = 0 $ \\

\textbf{Пр. 1}
\[Bu = -\frac{ d^2 }{d x^2 }u \qquad \textrm{в } L_2 (0,1); \qquad D_B = \{u \in C^2_0 (0,1): u(0) = u(1) = 0\} \]
\[ (B u, v) = - \int\limits_{0}^{1} v  \frac{d^2 u }{d x^2} dx = \int\limits_{0}^{1} \frac{du }{dx} \frac{d v }{d x } dx - v \left.\frac{d u }{dx }\right|^1_0 = - \int\limits_{0}^{1} u \frac{d^2v }{dx^2 } = (u, Bv) \quad \forall u,v \in D_B \]
\[ (Bu, u) = \int\limits_{0}^{1} {\left(\frac{du }{dx }\right)}^2 dx = 0 \]
\[ (Bu, u) = 0 \Rightarrow \frac{du }{dx } = 0 \Rightarrow u = const, u(0) = 0 \Rightarrow u = 0 \]

\textbf{Пр. 2}
\[ Cu = - \frac{ d^2 }{dx^2 }u, \qquad D_C = \left\{ u \in C^2(0,1),
\begin{cases}
	u'(0)+\alpha u(0)=0 \\
	u'(1)+\beta u(1)=0
\end{cases}
\alpha, \beta = const
\right\}
\]
\[(Cu, v) = \int\limits_{0}^{1} \frac{du}{dx} \frac{dv}{dx} dx + \alpha u(0)v(0) + \beta u(1)v(1) = (u, Cv)\]
\[\alpha > 0, \beta \geq 0\]
\[(Cu, u) = \int\limits_{0}^{1} {\left(\frac{du}{dx}\right)}^2 dx + \alpha u^2(0) + \beta u^2(1) \geq 0\]
\[\alpha = \beta = 0, \quad u \equiv 1 \Rightarrow (Cu, u) = 0 \Rightarrow C \text{ не является положительным}\]

\textbf{Пр. 3}
\[ Au = - \Delta u , \qquad D_A = \{ u \in C^2(\Omega): \quad u|_s = 0, \quad \Omega \subset \mathbb{R}^m, S = \partial \Omega, H = L_2(\Omega) \} \]
\[ (Au, u) = (-\Delta u, u ) = - \int\limits_{\Omega }^{} u \Delta u d \Omega = \int\limits_{\Omega }^{} \sum_{i = 1}^{m } {\left(\frac{\partial u }{\partial x_i }\right)}^2 d\Omega - \cancelto{0}{\int\limits_{S }^{} u \frac{\partial u }{\partial n } dS}  \geq 0 \]
\[ \frac{\partial u }{\partial x_i } = const, \ u|_s = 0 \Rightarrow u = 0 \]

\newpage

Рассмотрим мембрану

$\Omega$ в плоскости $(x,y)$, $\ u(x,y)$ --- изгиб мембраны
\[ - \Delta u = \frac{q}{T} \]

$q$ --- поперечная нагрузка на единицу площади

$T$ --- натяжение мембраны

$ {u|}_S = 0 $ --- мембрана закреплена на краях
\[ (A u , u) = (- \Delta u , u) = \iint\limits_{\Omega} \left[ \left(\frac{\partial u }{\partial x }\right)^2 + \left(\frac{ \partial u }{\partial y }\right)^2 \right] dx dy  \]

\subsection{Положительно определенные операторы}

\textbf{Опр.} Симметричный оператор $A$ называется положительно определенным, если
\begin{equation}
	\exists \gamma > 0 : (Au, u) \geq \gamma^2 {\|u\|}^2
\end{equation} \\
\textbf{Пр. 1 (продолжение)}
\[ B: u(0) = 0, u \in D_B \]
\[ u(x) = \int\limits_{0}^{x } u'(t) dt, \quad x \in [0, 1] \]
\[ u^2(x) \leq \int\limits_{0}^{x} 1^2 dt \cdot \int\limits_{0}^{x} {(u'(t))}^2 dt = x \int\limits_{0}^{x} {(u'(t))}^2 dt \leq x \int\limits_{0}^{1} {(u'(t))}^2 dt \]
\[ \int\limits_{0}^{1} u^2 (x) dx \leq \frac{1}{2} \int\limits_{0}^{1} {(u'(t))}^2 dt \]
\[ \gamma^2 \|u\|^2 \leq (Bu, u) , \quad \gamma = \sqrt{2} \quad \Rightarrow B \text{ является положительно определенным} \] \\
\textbf{Пр. 4}
\[ Lu = - \frac{d }{ d x } \left(x^3 \frac{du}{dx}\right) \quad \text{в } L_2 (0,1) \]
\[ D_L = \{u \in C^2[0,1], \ u(1) = 0 \} \]
\[(Lu, v) - (u, Lv) = \int\limits_{0}^{1} \frac{d}{dx} \left[ x^3 \left(u \frac{dv }{dx } - v \frac{du }{dx } \right)  \right] dx = \left. \left[ x^3 \left(u \frac{dv }{dx } - v \frac{du }{dx } \right) \right] \right|_0^1 = 0\]
\[ (Lu, u) = \int\limits_{0}^{1} x^3 {\left(\frac{du }{dx }\right)}^2 dx \geq 0 \quad \Rightarrow L \text{ является положительно определенным} \]
\[ \frac{(Lu, u)}{\|u\|^2} \geq \gamma^2, \qquad u_\delta (x) = \begin{cases}
	(\delta - x)^3, & 0 \leq x \leq \delta \\
	0, & \delta \leq x \leq 1
\end{cases}, \qquad u_\delta \in \mathcal{D}_L  \]
\[ \frac{(Lu_\delta , u_\delta)}{{\|u_\delta\|}^2}  = \frac{\int_{0}^{1} x^3 {(\frac{du_\delta}{dx})}^2 dx}{\int_{0}^{\delta} {(\delta -x)}^3 dx} = \frac{9 \int_{0}^{1} x^3 {(\delta -x)}^4 dx}{\int_{0}^{\delta} {(\delta -x)}^6 dx} = \frac{9}{40} \delta \quad \Rightarrow L \text{ не явл. положительно опр.} \]

\newpage

\subsection{Энергетическая норма}
Пусть $A$ --- положительно определен в $H$ (гильберт.) \\
На $D_A: \quad {[u, v]}_A = {(A u, v)}_H$ \\
Можно показать что выполняются все аксиомы скалярного произведения

\begin{enumerate}
	\item $ {[u, v]}_A = \overline{[v, u]}_A $ \\
	$ (Au, v) = (u, Av) = \overline{(Av, u)} = \overline{[v, u]} $
	\item $ [a_1 u + a_2 u, v] = a_1[u, v] + a_2[u, v]$
	\item $ (Au, u) = [u, u] \geq \gamma \|u\|^2 \geq 0 $
	\item $ [u, u] = 0 \Leftrightarrow u = 0 $
\end{enumerate}

$ |u| = [u, u] $ --- энергетическая норма

$ D_A $ предгильбертово, дополним его по $ {|\boldsymbol{\cdot}|}_A \ \Rightarrow \ $ гильбертово пр-во $ H_A $

\[ u \in H_A \Leftrightarrow \left[ \begin{array}{l}
	u \in D_A \\
	\exists u : \ \{ u_n \} \in D_A: \ | u_n - u | \underset{n \rightarrow \infty}{\rightarrow} 0
\end{array} \right. \]

\newpage
