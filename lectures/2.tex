\section{Лекция 2}

\subsection{Метод Бубнова -- Галеркина}

Рассмотрим приближенное решение в виде:
\[
w_n = \alpha_1 \varphi_1 + \alpha_2 \varphi_2 + \ldots + \alpha_n \varphi_n + \ldots
\]
где \(L\) и \(M\) --- дифференциальные операторы, задающие задачу:
\[
L w - \lambda M w = 0
\]
Приходим к системе уравнений вида:
\[
\sum\limits_{i=1}^{n} (A_{ik} - \lambda B_{ik}) \alpha_k = 0, \quad k = \overline{1,n}
\]
Чтобы решение было ненулевым, необходимо и достаточно, чтобы определитель равнялся нулю
\[
\begin{vmatrix}
	A_{11} - \lambda B_{11} & \dots & A_{1n} - \lambda B_{1n} \\
	\vdots & \ddots & \vdots \\
	A_{n1} - \lambda B_{n1} & \dots & A_{nn} - \lambda B_{nn}
\end{vmatrix}
= 0
\]
\[
N (x,y) = L w_n - \lambda M w_n \enspace \text{--- \enspace невязка}
\]
\[
N(x,y) \perp \varphi_i, \quad i = \overline{1,n}
\]

\subsection{Повторение}

\begin{enumerate}
	\item $ f(x) \overset{\text{п.в.}}{=} 0 \Rightarrow \int\limits_{\Omega} f(x) dx = 0 $
	\item $ \int\limits_{\Omega} f(x) dx = 0, \enspace f(x) \geq 0 \enspace \Rightarrow \enspace f(x)\overset{\text{п.в.}}{=} 0 $
	\item $ |f(x)| < \varphi(x), \varphi \text{ --- суммируема по Лебегу} \Rightarrow f(x) \text{ --- суммируема по Лебегу} $
	\item $ \{\varphi_n(x)\} $ --- суммируемы с квадратами по Лебегу
	\\ \\
	$\lim\limits_{n,k \rightarrow \infty} \int\limits_{\Omega}{|\varphi_k (x) - \varphi_{n} (x)|}^2 dx = 0
	$
	\\
\end{enumerate}

Обозначим $V$ -- линейное пространство \\

$ (\varphi , \psi) - \text{скалярное произведение:} \quad (\boldsymbol{\cdot},\boldsymbol{\cdot}): V \times V \rightarrow \mathbb{C} $

\begin{enumerate}
	\item $  (\varphi, \psi) = \overline{(\psi, \varphi)} $
	\item $ (a_1 \varphi_1 + a_2 \varphi_2, \psi) = a_1 (\varphi_1, \psi) + a_2 (\varphi_2, \psi) $
	\item $ (\varphi , \varphi) \geq 0 $
	\item $ (\varphi, \varphi) = 0 \quad \Rightarrow \quad \varphi = \mathbf{0} $
\end{enumerate}

$ \| \varphi \| = \sqrt{(\varphi, \varphi)} $ \\

\begin{itemize}
	\item Неравенство Коши-Буняковского
	
	$ | (\varphi, \psi) | \leq \| \varphi \| \| \psi \|$
	
	\item Неравенство треугольника
	
	$ \| \varphi + \psi \| \leq \| \varphi \| + \| \psi \| $
\end{itemize}
\[ L_2(\Omega): \quad (\varphi, \psi) = \int\limits_{\Omega}^{} \varphi(x) \overline{\psi(x)}dx \]
\[ L_2(\Omega , \sigma): \quad (\varphi, \psi) = \int\limits_{\Omega}^{} \varphi(x) \overline{\psi(x)}\sigma (x) dx \]
\[ L_2(\Omega^m): \quad (\varphi, \varphi) = \int\limits_{\Omega}^{} \sum_{k=1}^{m} \varphi_k(x) \overline{\varphi_k(x)}dx \] \\

\textbf{Критерий линейной зависимости системы функций}

\begin{gather*}
	\varphi_1, ..., \varphi_n \text{ линейно зависима (ЛЗ) в } H
	\\
	\hspace{20mm} \Updownarrow
	\\
	\begin{vmatrix}
		(\varphi_1, \varphi_1) & \dots & (\varphi_1, \varphi_n) \\
		\vdots & \ddots & \vdots \\
		(\varphi_n, \varphi_1) & \dots & (\varphi_n, \varphi_n)
	\end{vmatrix}
	= 0
\end{gather*} \\

\textbf{Опр.} $M$ --- плотно в $H$, если $ \forall p \in H$ и $\forall \varepsilon >0 \enspace \exists \varphi_n \in M: \| \varphi_n - \varphi \| < \varepsilon $. \\

$ C_0^{(\infty)} (\Omega) $ плотно в $ L_2(\Omega) $

$\quad \quad \quad \quad \quad \uparrow$

$ \forall \varepsilon > 0 : \quad \forall \varphi \in H \quad $
\begin{tabular}[t]{l}
	$ \exists \varphi_n^1 \in C_0^{(\infty)}(\Omega) : \quad \|\varphi_n^1 - \varphi\| < \varepsilon/2 $ \\
	$ \exists \varphi_n^2 \in C_0^{\infty} (\Omega) : \quad \|\varphi_n^2 - \varphi_n^1\| < \varepsilon/2 $ \\
	$...$
\end{tabular}

$ C_0^{(k)} (\Omega) $ плотно в $ L_2(\Omega) $ \\ \\

$ \{\varphi_n \} $ --- ортонормированная система (ОНС)

$ (\varphi_n, \varphi_m) = \delta _{nm} $

$ {\|\varphi\|}^2  = {\|\varphi_1\|}^2 +{\|\varphi_2\|}^2+ ... +{\|\varphi_n\|}^2 + ...$ \\


$ \{ \varphi_n \} $ полная в  $H$, если из $ (\varphi, \varphi_k) = 0 \enspace \forall k \in \mathbb{N} \quad \Rightarrow \quad \varphi = \mathbf{0} $


$ \forall \varphi \in H: \quad a_k = (\varphi, \varphi_k) - \textrm{ коэффициенты Фурье} $ \\


\textbf{Теор.} $H$ --- гильбертово, $\{\varphi_k\}$ --- полная ортонормированная система (ПОНС) \\ \\
$ \Rightarrow {\| \varphi \|}^2 = \sum\limits^{\infty}_{k=1} {|a_k|}^2  = \sum\limits^{\infty}_{k=1} {|(\varphi, \varphi_k)|}^2 $ --- равенство Парсеваля \\ \\

\textbf{Теор.} $ \exists a_k: \quad \sum\limits_{k=1}^{\infty} {|a_k|}^2 $ сходится, $\{\varphi_n\}$ --- ПОНС в $H$, тогда: \\ \\ $\sum\limits_{k=1}^{\infty} a_k \varphi_k$ сходится по $\|\boldsymbol{\cdot}\|$ к $\varphi \in H$, при этом $\|\varphi\| = \sum\limits_{k=1}^{\infty} {|a_k|}^2$. \\ \\


\textbf{Опр.} $H$ cепарабельно если $ \exists M - $ счетное мн-во плотное в H. \\

\textbf{Теор.} $H$ сепарабельно $ \Leftrightarrow \exists $ ПОНС (счетная или конечная) в $H$. \\

$ \{ u: \int\limits_{\Omega}^{} u dx = 0 \} $ --- пример подпространства в $ L_2(\Omega) $. \\ \\

Пусть $ H_1 $ --- подпространство в $H$

$ \forall \varphi \in H \quad \exists ! \varphi_1 \in H_1: \| \varphi - \varphi_1 \| = \underset{\psi \in H_1}{\min}  \| \varphi - \psi \| $ --- проекция $\varphi$ на $H_1$

$ \varphi = \varphi_1 + \varphi_2 $, \quad \quad $ H_2 = \varphi \perp H_1 $ --- ортогональное дополнение \\

$l$ --- линейный функционал $: \quad M \subset H \rightarrow \mathbb{R}/\mathbb{C} $

$ |l_{\varphi}| \leq \|l\| \cdot {\|\varphi\|}_H $

$ \underset{\psi \rightarrow \varphi}{\lim} l_\psi = l_\varphi \quad \quad \quad \forall \varepsilon > 0 \quad \exists \delta: \| \psi - \varphi \| < \delta: \quad |l_\psi - l_\varphi| < \varepsilon$ \\ \\

\textbf{Теор. (Рисса)} $ \forall l $ --- непрерывного линейного функционала в $H$ $\exists! \psi \in H:  l_\varphi = (\varphi, \psi)$ \\ \\


Пусть $M$ --- плотно в $H$, \quad $\Phi: M \times M \rightarrow \mathbb{C} (\mathbb{R})  $

$ \Phi(\varphi, \psi): \Phi(\varphi, \psi) = \overline{\Phi(\psi, \varphi)} $

$ \Phi(\varphi, \varphi) $ --- квадратичная форма \\ \\


$ H: D_A \subset H $ --- область определения некоторого оператора А

Линейный оператор $A$ ограничен $\Leftrightarrow A$ непрерывен

$\varphi \in D_A, \quad A\varphi \in R_A$ --- область значений оператора $A$

$\varphi \in D_A \rightarrow ! \enspace A\varphi \in R_A$

\newpage