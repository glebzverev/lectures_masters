\section{Лекция 14}

\subsection{Вариационная постановка задачи на собственные значения для симметрично положительного оператора}

% Задача на собственные значения:
Рассмотрим спектральную задачу:

\begin{equation}
	A \varphi = \lambda \varphi, \qquad \varphi \in D(A) \subset H \label{14.1} \tag{14.1}
\end{equation}

где:
\begin{itemize}
	\item $A$ --- симметричный оператор в гильбертовом пространстве $H$
	\item $\lambda$ --- собственные значения (вещественные для симметричного оператора)
	\item $\varphi$ --- собственные функции
\end{itemize}

% Свойства собственных функций симметричного оператора:
\textbf{Свойство ортогональности:} Если $\lambda_1 \neq \lambda_2$ --- два различных собственных значения оператора $A$, то соответствующие собственные функции $\varphi_1$ и $\varphi_2$ ортогональны:

\[ (\varphi_1, \varphi_2) = 0 \]

% Связь собственного значения с квадратичной формой:
Для собственной функции $\varphi$, удовлетворяющей $A\varphi = \lambda \varphi$, имеем:

\[ (A\varphi, \varphi) = \lambda (\varphi, \varphi) \]

откуда:

\begin{equation}
	\lambda = \frac{(A\varphi, \varphi)}{(\varphi, \varphi)} \label{eq:14_2}
\end{equation}

% Определение оператора, ограниченного снизу:
\textbf{Определение.} Оператор $A$ называется \textbf{ограниченным снизу}, если для всех $\varphi \in D(A)$ верно:

\begin{equation}
	(A\varphi, \varphi) \geq k(\varphi, \varphi), \quad k \in \mathbb{R} \label{14.2} \tag{14.2}
\end{equation}

где $k$ --- некоторая (не обязательно положительная) константа.

% Существование инфимума:
Если $A$ ограничен снизу, то существует инфимум (точная нижняя грань) отношения Рэлея:

\[ \frac{(A \varphi, \varphi)}{(\varphi, \varphi)} \geq k \quad \Rightarrow \quad \exists \, d = \inf_{\varphi \in D(A)} \frac{(A \varphi, \varphi)}{(\varphi, \varphi)} \geq k \]

% Определение функционала Рэлея:
Функционал:

\begin{equation}
	F(\varphi) = \frac{(A \varphi, \varphi)}{(\varphi, \varphi)} \label{14.4} \tag{14.4}
\end{equation}

называется \textbf{функционалом Рэлея} (Rayleigh quotient).

% ОСНОВНАЯ ТЕОРЕМА 1 О МИНИМАЛЬНОМ СОБСТВЕННОМ ЗНАЧЕНИИ:

\textbf{Теорема 1 (о минимальном собственном значении).} 

Пусть $A$ --- симметричный оператор, ограниченный снизу, и пусть

\[ d = \inf_{\varphi \in D(A)} \frac{(A \varphi, \varphi)}{(\varphi, \varphi)} \]

Если существует $\varphi_0 \neq 0 \in D(A)$ такой, что

\[ \frac{(A \varphi_0, \varphi_0)}{(\varphi_0, \varphi_0)} = d \]

то число $d$ является собственным значением оператора $A$, то есть $d = \lambda_1 \leq$ всех других собственных значений, и $\varphi_0$ --- соответствующая собственная функция:

\[ A \varphi_0 = d \varphi_0 \]

% Доказательство:
\textit{Доказательство:}

Возьмём произвольное $\eta \in D(A)$ такое, что при всех $t \in \mathbb{R}$ имеем $\varphi_0 + t \eta \in D(A)$ (например, если $D(A)$ выпукло, это верно для любого $\eta \in D(A)$).

Определим функцию одного переменного:

\[ \psi(t) = F(\varphi_0 + t \eta) = \frac{(A(\varphi_0 + t \eta), \varphi_0 + t \eta)}{(\varphi_0 + t \eta, \varphi_0 + t \eta)} \]

Разложим:

\begin{multline*}
	\psi(t) = \frac{t^2(A \eta, \eta) + 2t \operatorname{Re}(A \varphi_0, \eta) + (A \varphi_0, \varphi_0)}{t^2 (\eta, \eta) + 2t\operatorname{Re}(\varphi_0, \eta) + (\varphi_0, \varphi_0)}
\end{multline*}

Так как $\varphi_0$ доставляет минимум функционалу $F$, функция $\psi(t)$ достигает минимума при $t=0$:

\[ \psi'(0) = 0 \]

Вычисляя производную (используя правило для производной дроби) и приравнивая к нулю, получаем:

\[ \operatorname{Re} (A \varphi_0 - d \varphi_0, \eta) = 0 \quad \forall \eta \in D(A) \]

Заменяя $\eta$ на $i\eta$ (если $D(A)$ замкнуто относительно умножения на мнимую единицу), получаем:

\[ \operatorname{Im} (A \varphi_0 - d \varphi_0, \eta) = 0 \quad \forall \eta \in D(A) \]

Объединяя вещественную и мнимую части:

\[ (A \varphi_0 - d \varphi_0, \eta) = 0 \quad \forall \eta \in D(A) \]

Так как множество $D(A)$ плотно в $H$ (или мы выбираем его таким образом), то:

\[ A \varphi_0 - d \varphi_0 = 0 \quad \Rightarrow \quad A \varphi_0 = d \varphi_0 \]

Минимальность $d$ следует из того, что для любого собственного значения $\lambda_i$ оператора $A$:

\[ \lambda_i = \frac{(A \varphi_i, \varphi_i)}{(\varphi_i, \varphi_i)} \geq d \]

по определению $d$ как инфимума. $\square$

% ТЕОРЕМА 2 О ХАРАКТЕРИЗАЦИИ ПОСЛЕДОВАТЕЛЬНЫХ СОБСТВЕННЫХ ЗНАЧЕНИЙ:

\textbf{Теорема 2 (о последовательных собственных значениях).}

Пусть $\lambda_1 \leq \lambda_2 \leq \ldots \leq \lambda_n$ --- первые $n$ собственных значений симметричного ограниченного снизу оператора $A$, и $\varphi_1, \varphi_2, \ldots, \varphi_n$ --- соответствующие ортонормированные собственные функции.

Если существует функция $\varphi_{n+1} \neq 0$ такая, что

\begin{equation}
	\varphi_{n+1} = \operatorname{argmin}_{\substack{\varphi \in D(A) \\ (\varphi, \varphi_i) = 0, \, i=\overline{1,n}}} \frac{(A \varphi, \varphi)}{(\varphi, \varphi)} \label{14.6} \tag{14.6}
\end{equation}

(то есть минимизирует функционал Рэлея среди всех функций, ортогональных первым $n$ собственным функциям), то $\varphi_{n+1}$ является собственной функцией $A$, соответствующей собственному значению:

\[ \lambda_{n+1} = \frac{(A \varphi_{n+1}, \varphi_{n+1})}{(\varphi_{n+1}, \varphi_{n+1})} \]

% Идея доказательства:
\textit{Доказательство (идея):} Процедура аналогична Теореме 1. Любая вариация $\varphi_{n+1} + t\eta$, где $(\eta, \varphi_i) = 0$ для $i = \overline{1,n}$, сохраняет ортогональность. Условие минимума $\frac{d}{dt}\psi(t)|_{t=0} = 0$ даёт ортогональность $(A\varphi_{n+1} - \lambda_{n+1}\varphi_{n+1})$ всем таким вариациям. Так как подпространство, ортогональное $\{\varphi_1, \ldots, \varphi_n\}$, плотно на него, мы получаем $A\varphi_{n+1} = \lambda_{n+1}\varphi_{n+1}$. $\square$

\subsection{Обобщённая спектральная задача}

% Обобщённая задача на собственные значения:
Рассмотрим \textbf{обобщённую задачу на собственные значения}:

\begin{equation}
	A \varphi - \lambda B \varphi = 0 \label{eq:14_*}
\end{equation}

где:
\begin{itemize}
	\item $A$ и $B$ --- симметричные операторы
	\item $A$ ограничен снизу
	\item $B$ положительно определён (то есть $(B\varphi, \varphi) > 0$ для всех $\varphi \neq 0$)
	\item $D(A) \subset D(B) \subset H$
\end{itemize}

% Собственное значение в обобщённой задаче:
Собственное значение может быть выражено через отношение:

\[ \lambda = \frac{(A \varphi, \varphi)}{(B \varphi, \varphi)} \]

% ТЕОРЕМА О ОРТОГОНАЛЬНОСТИ В ОБОБЩЁННОЙ ЗАДАЧЕ:

\textbf{Теорема 3 (ортогональность в обобщённой задаче).}

Если $\lambda_k \neq \lambda_m$ --- два различных собственных значения обобщённой задачи, соответствующие собственным функциям $\varphi_k$ и $\varphi_m$, то:

\[ (B \varphi_k, \varphi_m) = 0 \]

Собственные функции ортогональны относительно скалярного произведения, определяемого оператором $B$ (это называется $B$-ортогональностью).

% ТЕОРЕМА О МИНИМУМЕ В ОБОБЩЁННОЙ ЗАДАЧЕ:

\textbf{Теорема 4 (о минимуме в обобщённой задаче).}

Пусть

\[ d = \inf_{\varphi \in D(A)} \frac{(A \varphi, \varphi)}{(B \varphi, \varphi)} \]

Если существует $\varphi_0 \neq 0$ такой, что 

\[ \frac{(A \varphi_0, \varphi_0)}{(B \varphi_0, \varphi_0)} = d \]

то число $d$ является минимальным собственным значением обобщённой задачи, а $\varphi_0$ --- соответствующая собственная функция.

% ТЕОРЕМА О ХАРАКТЕРИЗАЦИИ ПОСЛЕДОВАТЕЛЬНЫХ ЗНАЧЕНИЙ В ОБОБЩЁННОЙ ЗАДАЧЕ:

\textbf{Теорема 5 (последовательные собственные значения в обобщённой задаче).}

Пусть $\lambda_1 \leq \lambda_2 \leq \ldots \leq \lambda_n$ --- первые $n$ собственных значений обобщённой задачи с соответствующими $B$-ортонормированными собственными функциями $\varphi_1, \ldots, \varphi_n$ (то есть $(B\varphi_i, \varphi_j) = \delta_{ij}$).

Если существует функция $\varphi_{n+1}$ такая, что

\[ \lambda_{n+1} = \min_{\substack{\varphi \in D(A) \\ (B\varphi, \varphi_k) = 0, \, k=\overline{1,n}}} \frac{(A \varphi, \varphi)}{(B \varphi, \varphi)} \]

то $\varphi_{n+1}$ является собственной функцией, соответствующей собственному значению $\lambda_{n+1}$.

\subsection{Метод Ритца для задачи на собственные значения}

% Идея метода Ритца:
\textbf{Идея метода Ритца:} Вместо поиска минимума функционала Рэлея на всём пространстве $D(A)$, ищем минимум на конечномерном подпространстве, натянутом на выбранную систему базисных функций.

% Полнота системы функций:
Пусть система функций $\{\varphi_1, \varphi_2, \ldots\}$ полна в $H$, то есть для каждого $u \in D(A)$ и для любого $\varepsilon > 0$ существуют $n \in \mathbb{N}$ и коэффициенты $\alpha_1, \ldots, \alpha_n$ такие, что:

\[ \left\| u - \sum_{k=1}^{n} \alpha_k \varphi_k \right\| < \varepsilon \]

% Приближённое решение:
Рассмотрим приближённое решение в виде конечной линейной комбинации:

\begin{equation}
	u_n = \sum_{k=1}^{n} a_k \varphi_k \label{14_u_n}
\end{equation}

% Решение конечномерной задачи:
Выбираем коэффициенты $a_k$ так, чтобы минимизировать функционал Рэлея:

\[ (A u_n, u_n) = \sum_{k,m=1}^{n} (A \varphi_k, \varphi_m) a_k \overline{a_m} \to \min \]

при нормировке:

\begin{equation}
	(u_n, u_n) = \sum_{k,m=1}^{n} (\varphi_k, \varphi_m) a_k \overline{a_m} = 1 \label{14.14} \tag{14.14}
\end{equation}

% Применение метода множителей Лагранжа:
\textbf{Применение метода множителей Лагранжа:}

Составляем функцию Лагранжа:

\[ \mathcal{L} = (A u_n, u_n) - \lambda (u_n, u_n) \]

где $\lambda$ --- множитель Лагранжа (который окажется приближением к собственному значению).

Условие экстремума:

\[ \frac{\partial \mathcal{L}}{\partial a_m} = 0 \quad \Rightarrow \quad \sum_{k=1}^{n} a_k [(A \varphi_k, \varphi_m) - \lambda (\varphi_k, \varphi_m)] = 0, \quad m = \overline{1,n} \]

% Матричное уравнение:
Это можно записать в матричной форме:

\begin{equation}
	(G_A - \lambda G_B) \mathbf{a} = 0 \label{14.15} \tag{14.15}
\end{equation}

где:
\[ (G_A)_{km} = (A \varphi_k, \varphi_m), \quad (G_B)_{km} = (\varphi_k, \varphi_m) \]

% Условие существования нетривиального решения:
Для существования нетривиального решения $\mathbf{a} = (a_1, \ldots, a_n)^T$ необходимо и достаточно, чтобы:

\begin{equation}
	\det(G_A - \lambda G_B) = 0 \label{14.16} \tag{14.16}
\end{equation}

% Специальный случай --- ортонормированная система:
\textbf{Частный случай: ортонормированная система.}

Если система $\{\varphi_1, \ldots, \varphi_n\}$ уже ортонормирована (то есть $(\varphi_k, \varphi_m) = \delta_{km}$), то матрица Грама $G_B$ становится единичной:

\[ G_B = I \]

и уравнение упрощается:

\begin{equation}
	\det(G_A - \lambda I) = 0 \label{14.17} \tag{14.17}
\end{equation}

Это характеристическое уравнение матрицы $G_A$. Его решением являются собственные значения матрицы $G_A$.

% Структура уравнения на собственные значения:
Раскладывая определитель, получаем полиномиальное уравнение $n$-й степени относительно $\lambda$:

\[ \det(G_A - \lambda I) = (-1)^n \lambda^n + \text{члены меньших степеней} = 0 \]

Уравнение имеет ровно $n$ корней (считая кратности) в $\mathbb{C}$. Для симметричного оператора $A$ матрица $G_A$ симметрична, поэтому все корни вещественны.

% Восстановление собственных функций приближения:
\textbf{Нахождение приближённых собственных функций:}

Пусть $\lambda_j$ ($j = 1, \ldots, n$) --- один из корней уравнения. Подставляя $\lambda = \lambda_j$ в систему \eqref{14.15}, находим нетривиальное решение $(a_1^{(j)}, \ldots, a_n^{(j)})^T$.

Приближённая собственная функция:

\[ u_n^{(j)} = \sum_{k=1}^{n} a_k^{(j)} \varphi_k \]

После нормировки (так чтобы $(u_n^{(j)}, u_n^{(j)}) = 1$), приближённое собственное значение равно:

\[ \lambda_j^{(n)} = (A u_n^{(j)}, u_n^{(j)}) \]

% Сходимость метода Ритца:
\textbf{Свойства приближений метода Ритца:}

\begin{enumerate}
	\item \textbf{Монотонность:} Приближённые собственные значения $\lambda_j^{(n)}$ сходятся к точным собственным значениям $\lambda_j$ \textit{снизу} (то есть $\lambda_j^{(n)} \geq \lambda_j$ для всех $n$).
	
	\item \textbf{Сходимость:} При увеличении числа базисных функций приближения сходятся к точным значениям и функциям.
	
	\item \textbf{Оптимальность в подпространстве:} Для фиксированного $n$ приближение $u_n$ оптимально в том смысле, что минимизирует функционал Рэлея среди всех функций вида $\sum_{k=1}^n a_k \varphi_k$.
\end{enumerate}

% Практическое вычисление:
\textbf{Практическое вычисление:}

Алгоритм метода Ритца состоит из следующих шагов:

\begin{enumerate}
	\item Выбрать систему $n$ базисных функций $\varphi_1, \ldots, \varphi_n$ (обычно функции, удовлетворяющие однородным граничным условиям).
	
	\item Вычислить элементы матриц $G_A$ и $G_B$:
	\[ (G_A)_{km} = (A \varphi_k, \varphi_m), \quad (G_B)_{km} = (\varphi_k, \varphi_m) \]
	
	\item Решить задачу на собственные значения:
	\[ \det(G_A - \lambda G_B) = 0 \]
	
	\item Для каждого собственного значения $\lambda_j$ найти соответствующий собственный вектор и вычислить приближённую собственную функцию:
	\[ u_j^{(n)} = \sum_{k=1}^{n} a_k^{(j)} \varphi_k \]
\end{enumerate}

% Практическое значение метода Ритца:
Метод Ритца --- один из фундаментальных методов вычислительной математики. На его основе построены:
\begin{itemize}
	\item Метод конечных элементов (МКЭ) для задач на собственные значения
	\item Методы невязки и подпространственных итераций
	\item Алгоритмы нахождения нескольких собственных значений матриц
\end{itemize}

\newpage
