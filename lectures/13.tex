\section{Лекция 13}

% Рассмотрим одномерную задачу Штурма-Лиувилля с однородными граничными условиями Дирихле
\[ \left\{ \begin{array}{l}
	-\cfrac{d}{dx} \left[ p(x) \cfrac{du}{dx}\right] + q(x) u = f \\
	u(a) = u(b) = 0
\end{array} \right. \]

% Основная оценка аппроксимации в МКЭ:
% $u$ --- точное решение, $u_h$ --- приближённое решение МКЭ, $v_h$ --- произвольная функция из пространства МКЭ
\[ {\|u - u_h\|}_A \leq \inf_{v_h \in H_{A}^{(N)}} {\|u - v_h\|}_A, \quad \text{где} \ v_h = \Sum_{i=1}^{N-1} b_i \varphi_i; \quad \forall v_h \in H_{A}^{(N)} = \accentset{\circ}{W}_2^{1,h} \]

% Норма в энергетическом пространстве эквивалентна норме Соболева $W_2^1$:
\[ c_0 {\|u\|}_{{W}_2^1} \leq {\|u\|}_A \leq c_1 {\|u\|}_{{W}_2^1 \]

% Ошибка в норме $W_2^1$ не превосходит ошибку в энергетической норме с константой:
\[ {\|u-u_h\|}_{{W}_2^1} \leq c {\|u - v_h\|}_{{W}^1_2} \]

% Применяя инфимум, получаем оценку через наилучшее приближение:
\[ {\|u-u_h\|}_{{W}_2^1} \leq c \underset{v_h \in H_{A}^{(N)}}{\inf} {\|u - v_h\|}_{{W}^1_2} \]

% Если решение $u$ достаточно гладкое ($u \in W_2^2$), то наилучшее приближение кусочно-линейными функциями оценивается через шаг сетки $h$ и норму второй производной:
\[ {\|u-u_h\|}_{{W}_2^1} \leq ch {\|u\|}_{{W}_2^2} \leq ch {\|f\|}_{H=L_2} \]

Следовательно (по регулярности):
\[ {\|u-u_h\|}_{{W}_2^1} \leq ch {\|f\|}_H \]

% В норме $L_2$ ошибка имеет порядок $O(h^2)$ благодаря неравенству Фридрихса:
\[ {\|u-u_h\|}_{H=L_2} \leq c_2 h^2 {\|f\|}_H \qquad \text{(в силу неравенства Фридрихса)} \]

% Вариационная формулировка точной задачи:
\[ {[u, v]}_A = (f, v), \quad \forall v \in \accentset{\circ}{W}_2^{1} \]

% Вариационная формулировка приближённой задачи МКЭ (проекция на подпространство):
\[ {[u_h, v_h]}_A = (f, v_h), \quad  \forall v_h \in \accentset{\circ}{W}_2^{1, h} = H_A^{(N)} \]

% Это означает, что ошибка ортогональна пространству МКЭ:
\[ [u - u_h, v_h] = 0 \quad \forall v_h \in \accentset{\circ}{W}_2^{1, h} \]

% Получение оценок второго порядка методом двойственности (Nitsche trick):
\textbf{Метод двойственности (Nitsche trick) для оценки $L_2$ нормы ошибки:}

Рассмотрим вспомогательную задачу $A \Phi = F$, где $F = u - u_h$:

По свойствам оператора $A$ (положительная определённость и регулярность) существует единственное решение $\Phi$:
\[ {\|\Phi\|}_{W^2_2} \leq c \|F\| = c \|u - u_h\| \]

Функция $\Phi$ удовлетворяет вариационному уравнению:
\[ [\Phi, v] = (F, v) \quad \forall v \in \accentset{\circ}{W}_2^{1} \]

Подставляя $v = u - u_h$:
\begin{multline*}
	(F, v) = (u - u_h, u - u_h) = (A \Phi, u - u_h) = [\Phi, u - u_h] - [u - u_h, \Phi_h] \\
	= [\Phi - \Phi_h, u - u_h] \leq {\|\Phi - \Phi_h\|}_A \cdot {\|u - u_h\|}_A \\
	\leq ch {\|f\|} \cdot {\|\Phi - \Phi_h\|}_A \leq \tilde{c} h {\|f\|} \cdot {\|\Phi - \Phi_h\|}_{W_2^1}
\end{multline*}

где $\Phi_h$ --- аппроксимация $\Phi$ из пространства МКЭ, выбираемая так, чтобы:
\[ {\|\Phi - \Phi_h\|}_{{W}_2^1} \leq \tilde{c} h \ {\|\Phi\|}_{W^2_2} \]

Подставляя оценки:
\[ {\|u-u_h\|}^2 \leq ch {\|f\|} \cdot {\|\Phi- \Phi_h\|}_{{W}_2^1} \leq ch^2 {\|f\|} \cdot {\|\Phi\|}_{W_2^2} \leq ch^2 {\|f\|} \cdot \|u - u_h\| \]

Сокращая на $\|u - u_h\|$:
\[ \|u - u_h\| \leq ch^2 {\|f\|} \]

% Рассмотрим смешанные граничные условия (одна граница --- Дирихле, другая --- Неймана):
\subsection{Смешанные граничные условия}

\[ \left\{ \begin{array}{l}
	-\cfrac{d}{dx} \left(p(x) \cfrac{du}{dx}\right) + q(x) u = f(x) \\
	u(a) = 0, \quad \cfrac{du}{dx}(b) = 0
\end{array} \right. \]

При этом приближённое решение имеет вид:
\[ u_h = \Sum_{i=1}^N a_i \varphi_i \]

где базисные функции $\varphi_i$ выбираются так, чтобы удовлетворять условию Дирихле $u_h(a) = 0$.

\textbf{Упр.} Проверить аналогичные оценки ошибки в норме $W_2^1$ для этого случая.

\textbf{Упр.} Найти матрицу системы $\hat{A}$ при равномерной сетке $h_i = h, \ i = \overline{1, N}$, постоянных коэффициентах $p = const$, $q = const$.

\subsection{Применение вариационного метода к задаче Дирихле для уравнения Лапласа}

% Рассмотрим двумерную задачу в прямоугольнике:
Рассмотрим задачу Дирихле для уравнения Лапласа на прямоугольной области:

\[ \Omega = \{(x,y) : 0 < x < a, \ 0 < y < b\} \] 

\[ \left\{ \begin{array}{l}
	- \Delta u = f(x, y) \quad \text{в} \ \Omega \\
	u|_{\partial \Omega} = 0 \quad \text{на границе}
\end{array}  \right. \]

% Оператор и его область определения:
Оператор Лапласа:
\[ A u = - \Delta u \]

\[ D(A) = \{u \in W_2^2(\Omega), \ \ u|_{\partial \Omega} = 0 \}, \qquad f \in L_2(\Omega) \]

Оператор $A$ симметричен и положительно определён на пространстве $H_A = \accentset{\circ}{W}_2^1(\Omega)$.

% Регулярность решения:
Следовательно, для любой $f \in L_2(\Omega)$ существует единственное решение $u \in \accentset{\circ}{W}_2^1 \cap W_2^2(\Omega)$ с оценкой:
\[ {\| u \|}_{W_2^2} \leq c \, {\| f \|}_H \]

% Энергетическое пространство и скалярное произведение:
Энергетическое пространство и скалярное произведение:
\[ H_A = \accentset{\circ}{W}_2^1 (\Omega), \quad {(u, v)}_A = \Int_{\Omega}^{} \left(\frac{\partial u}{\partial x}\frac{\partial v}{\partial x}+\frac{\partial u}{\partial y}\frac{\partial v}{\partial y}\right)d\Omega\]

% Вариационная формулировка:
Вариационная формулировка:
\[ {[u, v]}_A = (f, v), \quad \forall v \in H_A \]

% Построение пространства конечных элементов:
\textbf{Метод конечных элементов на треугольной сетке:}

Покроем область $\Omega$ треугольной сеткой. Базисные функции $\varphi_{ij}(x,y)$ определяются как кусочно-линейные функции, равные 1 в узле $(x_i, y_j)$ и 0 во всех остальных узлах сетки.

% Приближённое решение как линейная комбинация базисных функций:
Приближённое решение представляется в виде:
\[ u_h = \Sum_{i=1}^{N_x -1} \Sum_{j=1}^{N_y - 1} a_{ij} \varphi_{ij} (x, y), \qquad H_A^{(N)} = \accentset{\circ}{W}_2^{1, h} \subset H_A = \accentset{\circ}{W}_2^1 \]

где суммирование ведётся только по внутренним узлам (граничные узлы имеют коэффициенты, равные нулю).

% Дискретная система:
Коэффициенты $a_{ij}$ находятся из системы линейных алгебраических уравнений:
\[ {[u_h, \varphi_{kl}]}_A = (f, \varphi_{kl}) \]

которая может быть записана в матричной форме:
\[ \hat{A} \mathbf{a} = \mathbf{f}, \qquad \hat{A} = \left( A_{ijkl} \right), \qquad A_{ijkl} = {[\varphi_{ij}, \varphi_{kl}]}_A \]

\[ \mathbf{a} = \left( a_{1,1}, a_{2,1}, \ldots, a_{N_x-1,1}, \ldots, a_{1, N_y-1}, \ldots, a_{N_x-1, N_y-1} \right) \]

Матрица $\hat{A}$ имеет ленточную структуру: большинство элементов равны нулю.

\textbf{Упр.} Вычислить элементы матрицы $A_{ijkl}$ явно для базисных функций на треугольной сетке.

% Оценки ошибки:
\textbf{Оценки ошибки МКЭ:}

\[ {\|u - u_h\|}_{W_2^1} \leq ch \, {\|u\|}_{W_2^2} \leq ch \, c(f) \, \|f\| \]

\[ \|u - u_h\|_{L_2} \leq ch^2 \, \|f\| \]

% Общий оператор с переменными коэффициентами:
Для общего оператора эллиптического типа:

\[ A u = - \sum_{k=1}^m \cfrac{d}{dx_k} \left(p(x) \cfrac{\partial u}{\partial x_k}\right) + q(x)u \]

\textbf{Упр.} Показать, что дискретная схема МКЭ даёт аппроксимацию порядка $O(h)$ в энергетической норме и $O(h^2)$ в $L_2$.

% Для нетривиальных сеток (не обязательно равномерных):
При произвольной триангуляции области обозначим через $h$ максимальный размер элемента, а через $\theta_0$ минимальный угол в треугольниках.

Тогда:
\[ {\|u-u_h\|}_{W_2^1(\Omega)} \leq c \, \frac{h}{\sin \theta_0} \|f\| \]

что показывает зависимость от качества триангуляции (острые углы приводят к увеличению ошибки).

% Общая форма решения:
Приближённое решение в общем случае:
\[ u_h = \Sum_{i=1}^{N} a_i \varphi_i (x, y) \]

где $N$ --- общее число узлов в области (включая граничные узлы с нулевыми коэффициентами).

\subsection{Подходы к решению неоднородной краевой задачи}

% Неоднородная задача --- на границе заданы ненулевые значения:
Рассмотрим неоднородную задачу Дирихле:

\[ \left\{
\begin{array}{l}
	- \Delta u = f \quad \text{в} \ \Omega \quad (\Omega \text{ выпуклая, } \partial \Omega \text{ гладкая}) \\
	u|_{\partial \Omega} = g, \qquad f \in L_2 (\Omega), \ g \in W^{3/2}_2 (\partial \Omega)
\end{array}
\right. \]

% Оценка регулярности решения:
Регулярность решения определяется оценкой:
\[ c_3 ({\|f\|}_{L_2} + {\|g\|}_{W_2^{3/2}}) \leq {\|u\|}_{W^2_2(\Omega)} \leq c_4 ({\|f\|}_{L_2} + {\|g\|}_{W_2^{3/2}}) \]

% Существует несколько подходов к решению такой задачи численно:

\raisebox{.5pt}{\textcircled{\raisebox{-.9pt} {1}}} \textbf{Сведение к однородным граничным условиям}

Предположим, что существует продолжение $\Phi \in D(A)$ функции $g$ в область $\Omega$:
\[ \Phi|_{\partial \Omega} = g \]

Введём новую функцию:
\[ v = u - \Phi \]

тогда она удовлетворяет однородной краевой задаче:
\[ \left\{ \begin{array}{l}
	- \Delta v = \tilde{f} = f + \Delta \Phi \quad \in L_2(\Omega) \\
	v|_{\partial \Omega} = 0
\end{array} \right. \]

Решение исходной задачи:
\[ u_h = v_h + \Phi \]

\textbf{Преимущество:} Применяем стандартный МКЭ для задачи с нулевыми граничными условиями.

\textbf{Недостаток:} Нужно найти подходящее продолжение $\Phi$, гладкое в $\Omega$.

\raisebox{.5pt}{\textcircled{\raisebox{-.9pt} {2}}} \textbf{«Снос» граничных условий (перенос с $\partial \Omega$ на $\partial \Omega_h$)}

Используем узлы сетки, расположенные на границе области, как неизвестные в системе МКЭ.

Обозначим узлы:
\[ \varphi_1, \ldots, \varphi_N \quad \text{--- внутренние узлы} \]
\[ \varphi_{N+1}, \ldots, \varphi_{\widetilde{N}} \quad \text{--- граничные узлы} \]

Приближённое решение:
\[ u_h = \Sum_{i=1}^{\widetilde{N}} a_i \varphi_i (x, y) \]

Система уравнений:

Для внутренних узлов ($i = \overline{1, N}$) применяем вариационное уравнение:
\[ {[u_h, \varphi_i]}_A = (f, \varphi_i) \]

Для граничных узлов ($i = \overline{N + 1, \widetilde{N}}$) задаём:
\[ a_i = u_h(x_i, y_i) = g(x_i, y_i) \]

\textbf{Преимущество:} Простая реализация, не требует явного продолжения $\Phi$.

\textbf{Недостаток:} Граничные условия точно удовлетворяются только в узлах на границе сетки $\partial \Omega_h$, а не на истинной границе $\partial \Omega$.

\raisebox{.5pt}{\textcircled{\raisebox{-.9pt} {3}}} \textbf{Метод штрафа (Penalty method)}

% Метод штрафа --- регуляризация неоднородной задачи:
Вместо исходной задачи рассмотрим модифицированную 3-ю краевую задачу (граничные условия Робена):

\[ \left\{
\begin{array}{l}
	- \Delta u_{\varepsilon} = f \quad \text{в} \Omega \\
	u_{\varepsilon} + \varepsilon \cfrac{\partial u_{\varepsilon}}{\partial n} = g \quad \text{на} \ \partial \Omega
\end{array}
\right., \qquad \varepsilon > 0 \ \text{мало} \]

где $\varepsilon$ --- параметр штрафа. При $\varepsilon \to 0$ решение $u_\varepsilon$ приближается к решению исходной задачи.

% Вариационная формулировка с штрафным членом:
Модифицированное энергетическое скалярное произведение с штрафным членом:

\[ {[u, v]}_A = \Int_{\Omega}^{} \left( \frac{\partial u}{\partial x} \frac{\partial v}{\partial x} + \frac{\partial u}{\partial y} \frac{\partial v}{\partial y} \right) d\Omega + \Int_{\partial \Omega}^{} \frac{1}{\varepsilon} u v \, dS \]

Приближённое решение:
\[ u_{\varepsilon, h} = \Sum_{i=1}^{N} a_i \varphi_i(x, y) \]

где все узлы (включая граничные) являются неизвестными, и коэффициенты $a_i$ находятся из системы:

\[ {[u_h, \varphi_i]}_A = (f, \varphi_i) + \Int_{\partial \Omega}^{} \frac{g}{\varepsilon} \varphi_i \, dS \]

% Оценки ошибки метода штрафа:
\textbf{Оценки ошибки МКЭ с методом штрафа:}

В норме $W_2^1$:
\[ {\|u_{\varepsilon} - u_{\varepsilon, h} \|}_{W_2^1 (\Omega)} \leq \frac{ch}{\sin \theta_0} \left( 1 + \frac{h}{\varepsilon} \right) \left({\|f\|}_{L_2} + \frac{1 }{\varepsilon} \| g \|_{W_2^{1/2} \left(\partial \Omega \right)} \right) \]

В норме $L_2$:
\[ {\|u_{\varepsilon} - u_{\varepsilon, h} \|}_{L_2 (\Omega)} \leq \frac{ch^2}{\sin^2 \theta_0} {\left( 1 + \frac{h}{\varepsilon} \right)}^2 \left({\|f\|}_{L_2} + \frac{1 }{\varepsilon} \| g \|_{W_2^{1/2} \left(\partial \Omega \right)} \right) \]

Ошибка регуляризации (разница между $u_\varepsilon$ и точным решением $u$):
\[ {\|u - u_{\varepsilon} \|}_{W_2^1 (\Omega)} \leq c \, \varepsilon \left({\|f\|}_{L_2} + \frac{1 }{\varepsilon} \| g \|_{W_2^{1/2} \left(\partial \Omega \right)} \right) \]

% Анализ метода штрафа:
\textbf{Анализ метода штрафа:}

\begin{itemize}
	\item При малых $\varepsilon$ штрафный член становится очень большим, что приводит к плохой обусловленности матрицы системы.
	
	\item Полная ошибка складывается из двух частей:
	\begin{enumerate}
		\item Ошибка аппроксимации МКЭ порядка $O(h)$ в $W_2^1$ или $O(h^2)$ в $L_2$
		\item Ошибка регуляризации порядка $O(\varepsilon)$
	\end{enumerate}
	
	\item Оптимальный выбор параметра: $\varepsilon \approx h$, тогда полная ошибка имеет порядок $O(h)$.
	
	\item \textbf{Преимущество:} Не требует явного продолжения граничных данных, универсальный метод.
	
	\item \textbf{Недостаток:} Параметр штрафа нужно подбирать, при малых $\varepsilon$ система плохо обусловлена.
\end{itemize}

\newpage
