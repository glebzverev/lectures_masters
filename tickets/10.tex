\section{Билет 10. Третья краевая задача для уравнения Штурма-Лиувилля}

\subsection*{Постановка третьей краевой задачи}

Третья краевая задача (граничные условия Робена) имеет вид:
\[
\boxed{\begin{cases}
-\frac{d}{dx}\left(p(x) \frac{du}{dx}\right) + q(x)u = f(x), & x \in (a, b) \\
u'(a) + \alpha u(a) = \varphi_a, \\
u'(b) + \beta u(b) = \varphi_b
\end{cases}}
\]

где $\alpha, \beta > 0$ --- коэффициенты, описывающие теплообмен или акустическое поглощение на границах.

\subsection*{Вариационная формулировка}

Умножим уравнение на пробную функцию $v$ и проинтегрируем:
\[
\int_a^b p(x) u'(x) v'(x) dx + \int_a^b q(x) u(x) v(x) dx + \alpha u(a)v(a) + \beta u(b)v(b)
\]
\[
= \int_a^b f(x) v(x) dx + \varphi_a v(a) + \varphi_b v(b)
\]

Здесь граничные члены из интегрирования по частям переходят в граничные члены вариационного уравнения.

\subsection*{Энергетическое пространство}

Для третьей краевой задачи энергетическое пространство:
\[
H_A = W_2^1(a, b)
\]

то есть без условия равенства нулю на границе (в отличие от задачи Дирихле).

Энергетическое скалярное произведение:
\[
[u, v]_A = \int_a^b p(x) u'(x) v'(x) dx + \int_a^b q(x) u(x) v(x) dx + \alpha u(a)v(a) + \beta u(b)v(b)
\]

\subsection*{Обобщённая формулировка}

\textbf{Определение:} Обобщённым решением третьей краевой задачи называется функция $u \in W_2^1(a, b)$ такая, что:
\[
[u, v]_A = (f, v)_{L_2} + \varphi_a v(a) + \varphi_b v(b) \quad \forall v \in W_2^1(a, b)
\]

\subsection*{Положительная определённость оператора}

\textbf{Теорема:}

Оператор третьей краевой задачи положительно определён:
\[
[u, u]_A \geq \gamma^2 \|u\|_{W_2^1}^2 \quad \text{для некоторого} \quad \gamma > 0
\]

\textit{Доказательство:}

\[
[u, u]_A = \int_a^b p(x)(u')^2 dx + \int_a^b q(x)u^2 dx + \alpha u^2(a) + \beta u^2(b)
\]

Все слагаемые неотрицательны, и если хотя бы $p_0 > 0$ и $\alpha, \beta > 0$, то оператор положительно определён.

\subsection*{Естественные краевые условия}

При выводе вариационной формулировки граничные условия третьего рода появляются \textbf{естественным образом} из интегрирования по частям и входят в определение энергетического скалярного произведения.

Граничные условия, которые появляются естественно при вариационной формулировке (а не требуются априори для функций из подпространства), называются \textbf{естественными граничными условиями}.

\textbf{Контраст:}

- \textbf{Существенные (главные) условия} (как Дирихле) требуют функции из пространства допустимых (задаются на уровне пространства)

- \textbf{Естественные условия} (как Робена) возникают из вариационной формулировки (следуют из интегрирования по частям)

\subsection*{Построение минимизирующей последовательности}

Метод Ритца для третьей краевой задачи:

1. Выбираем полную систему $\{\varphi_k\}$ в $W_2^1(a, b)$ (без требования нулевых граничных условий)

2. Ищем приближённое решение:
\[
u_n = \sum_{k=1}^{n} a_k \varphi_k(x)
\]

3. Из условия:
\[
[u_n, \varphi_j]_A = (f, \varphi_j) + \varphi_a \varphi_j(a) + \varphi_b \varphi_j(b)
\]

получаем систему для $a_k$

4. Минимизируем энергетический функционал:
\[
J(u) = [u, u]_A - 2(f, u)_{L_2} - 2[\varphi_a u(a) + \varphi_b u(b)]
\]

\subsection*{Примеры естественных условий в физике}

\textbf{Пример 1: Задача теплопроводности}

На границе задана температура окружающей среды и коэффициент теплопередачи:
\[
-k \frac{du}{dx}\bigg|_{x=L} = \alpha(u(L) - u_{\text{ext}})
\]

что эквивалентно:
\[
\frac{du}{dx}\bigg|_{x=L} + \frac{\alpha}{k} u(L) = \frac{\alpha}{k} u_{\text{ext}}
\]

\textbf{Пример 2: Акустическая задача}

На границе задано условие импеданса (связь давления и скорости на границе с поглощающим материалом):
\[
\frac{\partial p}{\partial n} + Z p = 0
\]

где $Z$ --- коэффициент импеданса.

\subsection*{Дополнительные пояснения}

\textbf{В.: Почему граничные условия третьего рода называют естественными?}

О.: Потому что они не требуют отбора функций на этапе выбора пространства. Они автоматически возникают как часть энергетического функционала при интегрировании по частям.

\textbf{В.: Как выбрать базисные функции для третьей краевой задачи?}

О.: Можно выбрать любые функции из $W_2^1$ без требования обращаться в нуль на границе. Например, степенные функции: $\varphi_k(x) = x^k$, $k = 0, 1, 2, \ldots$

\textbf{В.: В чём преимущество естественных условий?}

О.: Естественные условия упрощают численные схемы: не нужно явно включать граничные условия в пространство допустимых функций, они автоматически учитываются в энергетическом функционале.

\textbf{В.: Как связаны третья краевая задача и задачи Дирихле-Неймана?}

О.: Задача Дирихле --- частный случай при $\alpha \to \infty$ (жёсткое закрепление). Задача Неймана --- при $\alpha \to 0$ (свободная граница).
