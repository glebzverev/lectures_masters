\section{Билет 2. Энергетический метод для положительно определённых операторов в гильбертовом пространстве}

\subsection*{1. Положительно определённые операторы}

\subsubsection*{Определение и свойства}

Пусть $H$ --- гильбертово пространство со скалярным произведением $(\cdot, \cdot)$ и нормой $\|\cdot\| = \sqrt{(\cdot, \cdot)}$.

Линейный оператор $A: D_A \subset H \rightarrow H$ называется \textbf{положительно определённым}, если:
\begin{enumerate}
	\item $A$ --- самосопряжённый: $A = A^*$, т.е. $(A\varphi, \psi) = (\varphi, A\psi)$ для всех $\varphi, \psi \in D_A$
	\item $A$ --- положительный: $(A\varphi, \varphi) \geq 0$ для всех $\varphi \in D_A$
	\item $A$ --- строго положительный: существует $c > 0$ такое, что
	\[
	(A\varphi, \varphi) \geq c \|\varphi\|^2 \quad \forall \varphi \in D_A
	\]
\end{enumerate}

Константа $c$ называется \textbf{константой положительной определённости} или \textbf{констант}ой эллиптичности.

\subsubsection*{Энергетическое пространство}

Для положительно определённого оператора $A$ определим \textbf{энергетическое пространство} $H_A$ как пополнение $D_A$ по норме:
\[
\|\varphi\|_A = \sqrt{(A\varphi, \varphi)}
\]

Это норма, эквивалентная исходной:
\[
\sqrt{c} \|\varphi\| \leq \|\varphi\|_A \leq \sqrt{\|A\|} \|\varphi\|,
\]
где $\|A\|$ --- норма оператора $A$.

Скалярное произведение в $H_A$:
\[
[\varphi, \psi]_A = (A\varphi, \psi) = (\varphi, A\psi)
\]

\subsection*{2. Теорема о функционале энергии}

\subsubsection*{Вариационная задача}

Рассмотрим задачу минимизации функционала энергии. Пусть дана задача:
\[
A u = f, \quad u \in D_A, \quad f \in H
\]

Ей соответствует вариационная задача:
\[
J(v) = \frac{1}{2}(Av, v) - (f, v) \rightarrow \min_{v \in H_A}
\]

\subsubsection*{Основная теорема}

\textbf{Теорема.} Пусть $A$ --- положительно определённый оператор в гильбертовом пространстве $H$, $f \in H$. Тогда:

\begin{enumerate}
	\item Существует единственное решение $u \in H_A$ вариационной задачи:
	\[
	J(u) = \inf_{v \in H_A} J(v)
	\]
	
	\item Это решение удовлетворяет операторному уравнению:
	\[
	A u = f
	\]
	
	\item Минимальное значение функционала равно:
	\[
	J(u) = -\frac{1}{2}(f, u) = -\frac{1}{2}\|u\|_A^2
	\]
	
	\item Энергия функции $v$ может быть выражена как:
	\[
	\|v - u\|_A^2 = J(v) - J(u)
	\]
	то есть расстояние от $v$ до решения в энергетической норме равно избытку функционала энергии.
\end{enumerate}

\subsubsection*{Доказательство (идея)}

Функционал энергии --- это квадратичный функционал. Его первая вариация:
\[
\delta J(v; \psi) = (Av, \psi) - (f, \psi) = (Av - f, \psi)
\]

Для минимума необходимо $\delta J(u; \psi) = 0$ при всех $\psi \in H_A$:
\[
(Au - f, \psi) = 0 \quad \forall \psi \in H_A
\]

Поскольку $H_A$ плотно в $H$, отсюда следует $Au = f$.

Строгая положительная определённость гарантирует строгую выпуклость функционала и единственность минимума.

\subsection*{3. Примеры положительно определённых дифференциальных операторов}

\subsubsection*{Пример 1: Оператор Штурма-Лиувилля}

Рассмотрим на $H = L_2(a,b)$ оператор:
\[
A u = -\frac{d}{dx}\left(p(x) \frac{du}{dx}\right) + q(x)u
\]

с граничными условиями Дирихле: $u(a) = u(b) = 0$.

Здесь:
\begin{itemize}
	\item $p(x) \geq p_0 > 0$, $p \in C^1[a,b]$
	\item $q(x) \geq 0$, $q \in C[a,b]$
\end{itemize}

Оператор положительно определён с константой $c = p_0 > 0$:
\[
(Au, u) = \int_a^b \left[p(x)\left(\frac{du}{dx}\right)^2 + q(x)u^2\right] dx \geq p_0 \int_a^b \left(\frac{du}{dx}\right)^2 dx = p_0 \|u\|_{H^1}^2
\]

Энергетическое пространство $H_A = H_0^1(a,b)$ (соболевское пространство функций с нулевыми граничными условиями).

\subsubsection*{Пример 2: Оператор Лапласа}

На области $\Omega \subset \mathbb{R}^n$ с граничными условиями Дирихле: $u|_{\partial\Omega} = 0$,
\[
A u = -\Delta u = -\sum_{i=1}^n \frac{\partial^2 u}{\partial x_i^2}
\]

Положительно определён с константой $c = \lambda_1 > 0$, где $\lambda_1$ --- первое собственное значение оператора Лапласа.

\[
(Au, u) = \int_{\Omega} |\nabla u|^2 dx \geq \lambda_1 \int_{\Omega} u^2 dx
\]

Энергетическое пространство: $H_A = H_0^1(\Omega)$ (градиентная норма $\|\nabla u\|_{L_2}$).

\subsubsection*{Пример 3: Оператор упругой пластины}

Введённый в Лекции 1, оператор четвёртого порядка:
\[
A u = \Delta^2 u = \frac{\partial^4 u}{\partial x^4} + 2\frac{\partial^4 u}{\partial x^2 \partial y^2} + \frac{\partial^4 u}{\partial y^4}
\]

с граничными условиями жёсткой заделки: $u|_{\partial\Omega} = 0$, $\frac{\partial u}{\partial n}\Big|_{\partial\Omega} = 0$.

Положительно определён:
\[
(Au, u) = \int_{\Omega} (\Delta u)^2 d\Omega \geq c \|\nabla^2 u\|_{L_2}^2
\]

Энергетическое пространство: $H_A = H_0^2(\Omega)$ (норма вторых производных).

\subsection*{4. Применение к методу Бубнова-Галеркина}

Для положительно определённого оператора метод Бубнова-Галеркина становится методом Ритца (совпадают).

Ищем приближенное решение:
\[
u_n = \sum_{k=1}^n \alpha_k \varphi_k
\]

минимизируя функционал энергии:
\[
J(u_n) = \frac{1}{2}(Au_n, u_n) - (f, u_n) \rightarrow \min
\]

Условия минимума дают систему линейных уравнений:
\[
\sum_{k=1}^n (A\varphi_i, \varphi_k) \alpha_k = (f, \varphi_i), \quad i = \overline{1,n}
\]

или в матричной форме:
\[
\sum_{k=1}^n A_{ik} \alpha_k = B_i
\]

где $A_{ik} = (A\varphi_i, \varphi_k)$, $B_i = (f, \varphi_i)$.

Положительная определённость гарантирует:
\begin{itemize}
	\item Матрица $\{A_{ik}\}$ положительно определена, система имеет единственное решение
	\item Погрешность оценивается в энергетической норме:
	\[
	\|u - u_n\|_A^2 \leq J(u) - J(u_n) = \inf_{v \in H_n} J(v) - J(u_n)
	\]
\end{itemize}


