\section{Билет 15. Постановка задачи Дирихле для уравнения Лапласа. Обобщённое решение}

\subsection*{Постановка задачи Дирихле для уравнения Лапласа}

Задача Дирихле на области $\Omega \subset \mathbb{R}^m$ с гладкой границей $\partial\Omega$:
\[
\boxed{\begin{cases}
\Delta u = 0 & \text{в } \Omega \\
u|_{\partial\Omega} = g & \text{на границе}
\end{cases}}
\]

где $g: \partial\Omega \to \mathbb{R}$ --- заданная граничная функция.

\subsection*{Физическая интерпретация}

Уравнение Лапласа $\Delta u = 0$ описывает стационарное состояние в системах без источников:
- Распределение температуры в проводнике без источников тепла
- Электростатический потенциал в области без зарядов
- Давление в установившемся потоке вязкой жидкости

\subsection*{Принцип максимума}

\textbf{Теорема (принцип максимума):}

Гармоническая функция (решение $\Delta u = 0$) достигает своего максимума и минимума на границе области.

Следствие: если $g \geq 0$ на $\partial\Omega$, то $u \geq 0$ в $\Omega$.

\subsection*{Единственность классического решения}

\textbf{Теорема:}

Если граничные данные $g$ непрерывны, существует единственное классическое решение $u \in C^2(\Omega) \cap C(\overline{\Omega})$.

\textit{Доказательство:} Если $u_1$ и $u_2$ --- два решения, то $w = u_1 - u_2$ удовлетворяет:
\[
\Delta w = 0, \quad w|_{\partial\Omega} = 0
\]

По принципу максимума, $w = 0$. $\square$

\subsection*{Вариационная формулировка}

При формулировке в виде задачи Дирихле для уравнения Пуассона $-\Delta u = 0$ (то есть $f = 0$):

Умножим на пробную функцию $v \in \accentset{\circ}{W}_2^1(\Omega)$ (с нулевыми значениями на границе):
\[
\int_\Omega \nabla u \cdot \nabla v \, d\Omega = 0
\]

\subsection*{Энергетическое пространство}

\textbf{Определение:} Обобщённым решением задачи Дирихле для уравнения Лапласа называется функция $u$ такая, что:
\[
u|_{\partial\Omega} = g \quad \text{(в смысле следа)}
\]

и для вспомогательной функции $w = u - \Phi$ (где $\Phi$ --- гладкое продолжение граничных данных):
\[
[w, v]_A = 0 \quad \forall v \in \accentset{\circ}{W}_2^1(\Omega)
\]

\textbf{Обобщённое решение:}

$u \in W_2^1(\Omega)$ такой, что:
\[
\int_\Omega \nabla u \cdot \nabla v \, d\Omega = 0 \quad \forall v \in \accentset{\circ}{W}_2^1(\Omega)
\]

и $u - g \in \accentset{\circ}{W}_2^1(\Omega)$.

\subsection*{Связь с минимизацией энергии}

Решение задачи Дирихле минимизирует функционал энергии (Дирихле):
\[
D(u) = \int_\Omega |\nabla u|^2 d\Omega
\]

на множестве функций с граничными значениями $u|_{\partial\Omega} = g$.

\subsection*{Существование обобщённого решения}

\textbf{Теорема:}

Для любых граничных данных $g \in W_2^{1/2}(\partial\Omega)$ существует единственное обобщённое решение $u \in W_2^1(\Omega)$ задачи Дирихле.

\textit{Доказательство:} Сводится к однородным граничным условиям через вспомогательную функцию $\Phi$. Тогда $w = u - \Phi$ ищется в $\accentset{\circ}{W}_2^1$ из вариационного уравнения $[w, v]_A = -[\Phi, v]_A$ для всех $v \in \accentset{\circ}{W}_2^1(\Omega)$.

\subsection*{Регулярность решения}

\textbf{Теорема о регулярности:}

Если $\Omega$ --- выпуклая область с гладкой границей и $g \in W_2^1(\partial\Omega)$ (или выше), то обобщённое решение $u \in W_2^2(\Omega)$ и удовлетворяет уравнению Лапласа почти всюду:
\[
\Delta u = 0 \quad \text{п.в. в } \Omega
\]

\subsection*{Метод Ритца для уравнения Лапласа}

Приближённое решение ищется в виде $u_n = \Phi_0 + w_n$, где $\Phi_0$ --- гладкое продолжение граничных данных, а $w_n \in H_N$ (конечномерное подпространство $\accentset{\circ}{W}_2^1$):
\[
\int_\Omega \nabla w_n \cdot \nabla \varphi_k \, d\Omega = -\int_\Omega \nabla \Phi_0 \cdot \nabla \varphi_k \, d\Omega
\]

\subsection*{Дополнительные пояснения}

\textbf{В.: Почему уравнение Лапласа называют гармоническим?}

О.: Потому что решения (гармонические функции) обладают свойством, что значение в любой точке равно среднему значению по окружающей сфере. Это следует из принципа максимума.

\textbf{В.: Как определяется след функции на границе?}

О.: Для функции из $W_2^1(\Omega)$ существует оператор следа $\gamma_0: W_2^1(\Omega) \to L_2(\partial\Omega)$, который ставит в соответствие функции её граничные значения. Он определён почти всюду на границе.

\textbf{В.: Почему задача Лапласа имеет единственное решение, а задача Пуассона нет (без условий совместности)?}

О.: Потому что граничные условия Дирихле полностью определяют поведение функции. При $f = 0$ граничные условия уникально определяют гармоническую функцию внутри.

\textbf{В.: Зачем нужна вспомогательная функция $\Phi_0$?}

О.: Она расширяет граничные данные $g$ в область, чтобы можно было искать решение как сумму $\Phi_0$ и функции с нулевыми граничными значениями.
