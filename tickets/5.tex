\section{Билет 5. Теорема о сходимости минимизирующей последовательности для функционала энергии}

\subsection*{Определение минимизирующей последовательности}

Последовательность $\{u_n\} \subset H_A$ называется \textbf{минимизирующей} для функционала энергии $J(u) = [u, u]_A - 2(f, u)$, если:
\[
\lim_{n \to \infty} J(u_n) = \inf_{u \in H_A} J(u) = m
\]

\subsection*{Вспомогательное неравенство}

Для любых $u, v \in H_A$ и минимизирующей последовательности имеем:
\[
J(u_n) = [u_n, u_n]_A - 2(f, u_n)
\]

Пусть $u_*$ --- точка минимума (существует по теореме Банаха). Тогда:
\[
[u_* - u_n, u_* - u_n]_A = [u_*, u_*]_A - 2[u_*, u_n]_A + [u_n, u_n]_A
\]

\subsection*{Теорема о сходимости}

\textbf{Теорема:}

Пусть $A$ --- положительно определённый оператор в гильбертовом пространстве $H$, и пусть $\{u_n\}$ --- минимизирующая последовательность для функционала энергии:
\[
\lim_{n \to \infty} J(u_n) = \inf_{u \in H_A} J(u)
\]

Тогда существует единственный элемент $u_* \in H_A$ такой, что последовательность $\{u_n\}$ сходится к $u_*$ в энергетической норме:
\[
|u_n - u_*|_A \to 0 \quad \text{при} \quad n \to \infty
\]

и $u_*$ минимизирует функционал:
\[
J(u_*) = \inf_{u \in H_A} J(u)
\]

\subsection*{Доказательство}

\textit{Шаг 1: равномерная ограниченность}

Так как последовательность $J(u_n)$ сходится, она ограничена. Пусть $M = \sup_n J(u_n) + 1$. Тогда:
\[
[u_n, u_n]_A = J(u_n) + 2(f, u_n)
\]

По неравенству Коши-Буняковского:
\[
|(f, u_n)| \leq |f|_{A^{-1}} |u_n|_A
\]

где $|f|_{A^{-1}}$ --- норма в двойственном пространстве. Следовательно:
\[
[u_n, u_n]_A \leq M + 2|f|_{A^{-1}} |u_n|_A
\]

Так как $[u_n, u_n]_A = |u_n|_A^2$, получаем оценку:
\[
|u_n|_A^2 - 2|f|_{A^{-1}} |u_n|_A - M \leq 0
\]

откуда $\sup_n |u_n|_A < \infty$.

\textit{Шаг 2: слабая компактность}

В гильбертовом пространстве из ограниченности последовательности следует наличие слабо сходящейся подпоследовательности:
\[
u_{n_k} \rightharpoonup u_* \quad \text{в} \quad H_A
\]

\textit{Шаг 3: сильная выпуклость}

Для любых $u, v \in H_A$ верно (параллелограмм):
\[
|u - v|_A^2 = 2|u|_A^2 + 2|v|_A^2 - |u + v|_A^2
\]

Функционал $J$ является сильно выпуклым с константой $\gamma^2 > 0$ (из положительной определённости $A$).

\textit{Шаг 4: единственность и сильная сходимость}

Из сильной выпуклости следует, что если последовательность слабо сходится и её образы под $J$ сходятся к инфимуму, то последовательность сильно сходится.

Именно, если $u_{n_k} \rightharpoonup u_*$ и $J(u_{n_k}) \to m$, то:
\[
J(u_*) \leq \liminf_{k \to \infty} J(u_{n_k}) = m
\]

но с другой стороны, $J(u_*) = m$ (так как минимум единственен), откуда:
\[
|u_{n_k} - u_*|_A^2 = [u_{n_k}, u_{n_k}]_A - 2[u_{n_k}, u_*]_A + [u_*, u_*]_A \to 0
\]

\subsection*{Приложения к численным методам}

\textbf{Следствие для метода Ритца:}

Если построить последовательность конечномерных подпространств $H_n \subset H_A$ таких, что $H_n \to H_A$ (в смысле Хаусдорфа), то приближённые решения $u_n = \operatorname*{argmin}_{v \in H_n} J(v)$ образуют минимизирующую последовательность и сходятся к точному решению.

\subsection*{Оценка скорости сходимости}

Пусть решение $u_* \in H_A$ достаточно гладко. Тогда скорость сходимости зависит от аппроксимационных свойств подпространства $H_n$:
\[
|u_n - u_*|_A \leq C \inf_{v \in H_n} |u_* - v|_A
\]

\subsection*{Дополнительные пояснения}

\textbf{В.: Всегда ли минимизирующая последовательность сходится?}

О.: В полном гильбертовом пространстве со сильно выпуклым функционалом --- да. Сильная выпуклость обеспечивается положительной определённостью оператора $A$.

\textbf{В.: Почему важна единственность минимума?}

О.: Единственность гарантирует, что сходится вся последовательность, а не только подпоследовательность. Это критично для численных методов.

\textbf{В.: Как эта теорема применяется на практике?}

О.: Она обосновывает метод Ритца и его варианты: строим конечномерное подпространство и решаем задачу минимизации в этом подпространстве. Решение автоматически сходится к точному решению при уменьшении параметра дискретизации.
