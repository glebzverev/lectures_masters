\section{Билет 12. Построение обобщённого решения для уравнения Пуассона с условиями Дирихле}

\subsection*{Постановка задачи}

Задача Дирихле для уравнения Пуассона на области $\Omega \subset \mathbb{R}^m$:
\[
\boxed{\begin{cases}
-\Delta u = f(x) & \text{в } \Omega \\
u|_{\partial \Omega} = 0 & \text{на границе}
\end{cases}}
\]

где $\Delta u = \sum_{i=1}^{m} \frac{\partial^2 u}{\partial x_i^2}$ --- оператор Лапласа.

\subsection*{Вариационная формулировка}

Умножим уравнение на пробную функцию $v \in \accentset{\circ}{W}_2^1(\Omega)$ и проинтегрируем по области:
\[
-\int_\Omega v \Delta u \, d\Omega = \int_\Omega f v \, d\Omega
\]

Применяя формулу Грина и используя граничные условия $v|_{\partial\Omega} = 0$:
\[
\int_\Omega \nabla u \cdot \nabla v \, d\Omega = \int_\Omega f v \, d\Omega
\]

\subsection*{Энергетическое пространство и скалярное произведение}

Энергетическое скалярное произведение:
\[
[u, v]_A = \int_\Omega \nabla u \cdot \nabla v \, d\Omega = \sum_{i=1}^{m} \int_\Omega \frac{\partial u}{\partial x_i} \frac{\partial v}{\partial x_i} d\Omega
\]

Энергетическая норма:
\[
|u|_A = \sqrt{\int_\Omega |\nabla u|^2 d\Omega}
\]

Энергетическое пространство:
\[
H_A = \accentset{\circ}{W}_2^1(\Omega) := \{u \in W_2^1(\Omega) : u|_{\partial\Omega} = 0\}
\]

\subsection*{Обобщённая формулировка}

\textbf{Определение:} Обобщённым решением задачи Дирихле для уравнения Пуассона называется функция $u \in \accentset{\circ}{W}_2^1(\Omega)$ такая, что:
\[
\boxed{[u, v]_A = (f, v)_{L_2} \quad \forall v \in \accentset{\circ}{W}_2^1(\Omega)}
\]

\subsection*{Положительная определённость оператора}

\textbf{Теорема:}

Оператор Лапласа с условиями Дирихле положительно определён в пространстве $\accentset{\circ}{W}_2^1(\Omega)$.

\textit{Доказательство:}

\[
[u, u]_A = \int_\Omega |\nabla u|^2 d\Omega
\]

По неравенству Фридрихса для функций с нулевыми граничными условиями:
\[
\|u\|_{L_2(\Omega)}^2 \leq C_F^2 \|\nabla u\|_{L_2(\Omega)}^2
\]

где $C_F$ --- постоянная Фридрихса. Следовательно:
\[
[u, u]_A = \|\nabla u\|_{L_2}^2 \geq \frac{1}{C_F^2} \|u\|_{L_2}^2 =: \gamma^2 \|u\|_{L_2}^2
\]

\subsection*{Неравенство Фридрихса}

\textbf{Теорема (неравенство Фридрихса):}

Для функции $u \in \accentset{\circ}{W}_2^1(\Omega)$ (с нулевыми граничными условиями) выполняется:
\[
\boxed{\|u\|_{L_2(\Omega)} \leq C_F \|\nabla u\|_{L_2(\Omega)}}
\]

где константа Фридрихса $C_F$ зависит от размера и формы области $\Omega$.

Для прямоугольника $(0, a) \times (0, b)$: $C_F = \frac{\sqrt{a^2 + b^2}}{\pi}$

Для отрезка $(0, L)$: $C_F = \frac{L}{\pi}$

\subsection*{Существование и единственность}

\textbf{Теорема:}

Для любой правой части $f \in L_2(\Omega)$ существует единственное обобщённое решение $u \in \accentset{\circ}{W}_2^1(\Omega)$ задачи Дирихле.

\textit{Доказательство:}

По теореме Рисса об ограниченных функционалах. Функционал $(f, v)_{L_2}$ ограничен в $H_A$:
\[
|(f, v)_{L_2}| \leq \|f\|_{L_2} \|v\|_{L_2} \leq \|f\|_{L_2} C_F |v|_A
\]

Следовательно, существует единственный $u \in H_A$ такой, что $[u, v]_A = (f, v)$ для всех $v \in H_A$.

\subsection*{Эквивалентная минимизационная задача}

Обобщённое решение минимизирует функционал:
\[
J(u) = \int_\Omega |\nabla u|^2 d\Omega - 2\int_\Omega f u \, d\Omega \to \min
\]

на $\accentset{\circ}{W}_2^1(\Omega)$.

\subsection*{Регулярность решения}

\textbf{Теорема о регулярности:}

Если $\Omega$ --- выпуклая область с гладкой границей и $f \in L_2(\Omega)$, то обобщённое решение $u$ принадлежит $W_2^2(\Omega)$ и удовлетворяет уравнению почти всюду:
\[
-\Delta u = f \quad \text{п.в. в } \Omega
\]

\subsection*{Метод Ритца}

Приближённое решение строится в виде:
\[
u_n(x) = \sum_{k=1}^{n} a_k \varphi_k(x)
\]

где $\varphi_k$ --- полная система в $\accentset{\circ}{W}_2^1(\Omega)$ (например, собственные функции оператора Лапласа).

Система алгебраических уравнений:
\[
\sum_{k=1}^{n} A_{jk} a_k = b_j
\]

где $A_{jk} = [\varphi_k, \varphi_j]_A = \int_\Omega \nabla \varphi_k \cdot \nabla \varphi_j d\Omega$

\subsection*{Дополнительные пояснения}

\textbf{В.: Почему неравенство Фридрихса настолько важно?}

О.: Потому что оно устанавливает эквивалентность нормы $|\nabla u|_{L_2}$ и нормы $\|u\|_{L_2}$ в пространстве функций с нулевыми граничными условиями. Это гарантирует положительную определённость оператора.

\textbf{В.: Как неравенство Фридрихса зависит от области?}

О.: Константа Фридрихса тем меньше, чем «тоньше» область. Для очень вытянутой области константа велика, что означает слабую связь между $u$ и $\nabla u$.

\textbf{В.: Почему граничные условия называют существенными?}

О.: Потому что они требуют выбора пространства $\accentset{\circ}{W}_2^1$, функции из которого обращаются в нуль на границе. Это в отличие от естественных условий, которые возникают из вариационной формулировки.

\textbf{В.: Можно ли применить метод Ритца прямо в классе гладких функций?}

О.: Да, можно, если базисные функции достаточно гладки. Но определение энергетического пространства и обобщённого решения позволяет работать с менее гладкими функциями и обосновывает сходимость метода.
