\section{Билет 7. Теорема о сходимости метода Ритца}

\subsection*{Формулировка метода Ритца}

Пусть $\{H_n\}$ --- последовательность конечномерных подпространств, где $H_n = \text{span}\{\varphi_1, \ldots, \varphi_n\}$ с полной системой $\{\varphi_k\}$ в $H_A$.

Приближённое решение $u_n \in H_n$ определяется из системы:
\[
[u_n, v]_A = (f, v) \quad \forall v \in H_n
\]

\subsection*{Основная теорема о сходимости}

\textbf{Теорема:}

Пусть $A$ --- положительно определённый оператор в гильбертовом пространстве $H$, и пусть $\{\varphi_k\}$ --- полная система в $H_A$. Тогда для последовательности приближённых решений $\{u_n\}$, полученных методом Ритца, выполняется:

\[
|u_n - u_*|_A \to 0 \quad \text{при} \quad n \to \infty
\]

где $u_* \in H_A$ --- точное решение вариационной задачи $[u, v]_A = (f, v)$ для всех $v \in H_A$.

\subsection*{Доказательство}

\textit{Шаг 1: Ортогональность ошибки}

Из определения $u_n$:
\[
[u_n, \varphi_j]_A = (f, \varphi_j) \quad \forall j = 1, \ldots, n
\]

и определения $u_*$:
\[
[u_*, \varphi_j]_A = (f, \varphi_j) \quad \forall j = 1, \ldots, n
\]

получаем:
\[
[u_n - u_*, \varphi_j]_A = 0 \quad \forall j = 1, \ldots, n
\]

\textit{Шаг 2: Оптимальность приближения}

Ошибка $u_n - u_*$ ортогональна всем базисным функциям, а значит, и всему подпространству $H_n$:
\[
[u_n - u_*, v]_A = 0 \quad \forall v \in H_n
\]

Это означает, что $u_n$ является $H_A$-проекцией $u_*$ на подпространство $H_n$:
\[
|u_n - u_*|_A = \inf_{v \in H_n} |u_* - v|_A
\]

\textit{Шаг 3: Полнота системы}

Так как система $\{\varphi_k\}$ полна в $H_A$, для любого $\varepsilon > 0$ и для $u_* \in H_A$ существует $N$ и коэффициенты $c_1, \ldots, c_N$ такие, что:
\[
\left|u_* - \sum_{k=1}^{N} c_k \varphi_k\right|_A < \varepsilon
\]

\textit{Шаг 4: Сходимость}

Из оптимальности приближения и полноты системы:
\[
|u_n - u_*|_A = \inf_{v \in H_n} |u_* - v|_A \to 0 \quad \text{при} \quad n \to \infty
\]

\subsection*{Оценка скорости сходимости}

Скорость сходимости зависит от аппроксимационных свойств подпространства $H_n$:

\textbf{Оценка:}
\[
|u_n - u_*|_A \leq C_1 \inf_{v \in H_n} |u_* - v|_A
\]

Если решение $u_* \in H_A$ имеет достаточную гладкость, например $u_* \in W_2^k(\Omega)$, то при использовании кусочно-полиномиальных базисных функций степени $m-1$ на сетке с шагом $h$:
\[
\inf_{v \in H_n} |u_* - v|_A \leq C h^{\min(m, k-1)}
\]

\subsection*{Сходимость в исходной норме $H$}

Помимо сходимости в энергетической норме $|\cdot|_A$, часто требуется сходимость в исходной норме пространства $H = L_2(\Omega)$.

\textbf{Теорема (дуальность):}

Если оператор $A$ имеет хорошие свойства регулярности (например, если решение $Au = v$ для $v \in H$ принадлежит $W_2^2$), то из сходимости в энергетической норме следует более быстрая сходимость в $L_2$:
\[
\|u_n - u_*\|_H \leq C |u_n - u_*|_A \cdot |u_n - u_*|_A
\]

что может дать сходимость порядка $O(h^2)$ вместо $O(h)$.

\subsection*{Практическое значение}

Теорема о сходимости метода Ритца обосновывает его применение к различным классам задач:

\begin{enumerate}
\item Подтверждает корректность метода: приближённые решения стремятся к точному

\item Даёт оценки скорости сходимости, которые определяют требуемое число неизвестных

\item Позволяет выбирать базисные функции для достижения требуемой точности

\item Работает не только для классических, но и для обобщённых решений
\end{enumerate}

\subsection*{Дополнительные пояснения}

\textbf{В.: Как объяснить ортогональность ошибки простыми словами?}

О.: Если приближённое решение $u_n$ удовлетворяет вариационному уравнению на подпространстве $H_n$, то его ошибка в смысле энергетического скалярного произведения перпендикулярна всему этому подпространству. Это означает, что $u_n$ --- наилучшее приближение $u_*$ в подпространстве.

\textbf{В.: Почему полнота системы критична?}

О.: Если система не полна, то существует элемент, который нельзя приблизить никакой конечной линейной комбинацией базисных функций. Тогда метод не сходится.

\textbf{В.: Можно ли использовать неполные системы?}

О.: В специальных случаях да, но тогда метод может сходиться только к приближённому решению. На практике всегда требуется полнота для гарантии сходимости.
