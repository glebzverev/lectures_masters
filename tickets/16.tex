\section{Билет 16. Подходы к решению задачи Дирихле для эллиптического уравнения 2 порядка с неоднородными краевыми условиями}

\subsection*{Постановка неоднородной задачи Дирихле}

\[
\begin{cases}
-\sum_{i,j=1}^{m} \frac{\partial}{\partial x_i}\left(a_{ij}(x) \frac{\partial u}{\partial x_j}\right) + c(x) u = f(x) & \text{в } \Omega \\
u|_{\partial\Omega} = g & \text{на границе}
\end{cases}
\]

где $a_{ij}$ --- симметричные, положительно определённые коэффициенты, $g \neq 0$ --- неоднородные граничные данные.

\subsection*{Основные подходы}

\subsubsection{Подход 1: Сведение к однородным граничным условиям}

Предположим, существует гладкая функция $\Phi(x)$ такая, что $\Phi|_{\partial\Omega} = g$.

Положим $u = \Phi + w$, где $w|_{\partial\Omega} = 0$. Тогда:
\[
-\sum_{i,j} \frac{\partial}{\partial x_i}\left(a_{ij} \frac{\partial w}{\partial x_j}\right) + c w = \tilde{f} := f + \sum_{i,j} \frac{\partial}{\partial x_i}\left(a_{ij} \frac{\partial \Phi}{\partial x_j}\right) - c\Phi
\]

Получается задача с однородными граничными условиями для $w$.

\textbf{Преимущество:} используются стандартные методы для однородных условий

\textbf{Недостаток:} нужно найти подходящее продолжение $\Phi$

\subsubsection{Подход 2: Прямое решение с незави симыми граничными узлами}

При конечно-элементной аппроксимации:
- Внутренние узлы сетки: неизвестные функции
- Граничные узлы: задаём значение $u_h = g$ в узлах

Система имеет вид:
\[
\begin{pmatrix}
A_{II} & A_{IB} \\
A_{BI} & A_{BB}
\end{pmatrix}
\begin{pmatrix}
u_I \\
u_B
\end{pmatrix}
=
\begin{pmatrix}
f_I \\
f_B
\end{pmatrix}
\]

где индекс $I$ --- внутренние узлы, $B$ --- граничные узлы. Полагаем $u_B = g$, решаем:
\[
A_{II} u_I = f_I - A_{IB} g
\]

\textbf{Преимущество:} просто реализуется, не требует явного продолжения $\Phi$

\textbf{Недостаток:} граничные условия удовлетворяются только в узлах сетки

\subsubsection{Подход 3: Метод штрафа (Penalty method)}

Модифицируем функционал, добавляя штрафной член:
\[
J(u) = [u, u]_A - 2(f, u) + \frac{1}{\varepsilon} \int_{\partial\Omega} (u - g)^2 dS
\]

где $\varepsilon > 0$ --- малый параметр штрафа.

При $\varepsilon \to 0$ решение исходной задачи восстанавливается приблизительно.

\textbf{Преимущество:} не требует выбора $\Phi$, универсален

\textbf{Недостаток:} параметр $\varepsilon$ нужно подбирать, система плохо обусловлена при малых $\varepsilon$

\subsubsection{Подход 4: Метод Лагранжа (Lagrange multipliers)}

Вводим множитель Лагранжа $\lambda$ для условия $u|_{\partial\Omega} = g$:
\[
\mathcal{L}(u, \lambda) = [u, u]_A - 2(f, u) + 2\int_{\partial\Omega} \lambda(u - g) dS
\]

Условия оптимальности:
\[
[u, v]_A + \int_{\partial\Omega} \lambda v \, dS = (f, v) \quad \forall v \in W_2^1
\]
\[
\int_{\partial\Omega} \mu(u - g) dS = 0 \quad \forall \mu \in L_2(\partial\Omega)
\]

\textbf{Преимущество:} точно удовлетворяет граничные условия, даёт физическое значение множителя (тиск/сила на границе)

\textbf{Недостаток:} увеличивает размер системы

\subsubsection{Подход 5: Метод поднятия (Lifting)}

Явное построение функции $\Phi$ через решение вспомогательной задачи:
\[
\begin{cases}
-\Delta \Phi = 0 & \text{в } \Omega \\
\Phi|_{\partial\Omega} = g
\end{cases}
\]

(часто просто интерполяция граничных данных).

\textbf{Преимущество:} применимо для любых граничных данных

\textbf{Недостаток:} требует решения дополнительной задачи

\subsection*{Сравнение подходов}

\begin{tabular}{|l|c|c|c|}
\hline
\textbf{Подход} & \textbf{Реализация} & \textbf{Точность} & \textbf{Сложность} \\
\hline
Однородные условия & + & ++ & ++ \\
Прямой КЭ & ++ & + & + \\
Штраф & ++ & + & + \\
Лагранж & ++ & ++ & ++ \\
Поднятие & + & ++ & ++ \\
\hline
\end{tabular}

\subsection*{Рекомендации по выбору}

- Для простых областей и гладких $g$: сведение к однородным условиям
- Для сложных областей: прямой метод КЭ или штраф
- Для высокой точности: метод Лагранжа или поднятие
- На практике: часто комбинируют методы

\subsection*{Дополнительные пояснения}

\textbf{В.: Почему метод штрафа работает?}

О.: Штрафной член растёт как $(u-g)^2/\varepsilon$. При малом $\varepsilon$ нарушение условия $u = g$ становится очень дорогим, вынуждая минимизатор удовлетворять условию.

\textbf{В.: Как выбрать параметр штрафа?}

О.: Практически: $\varepsilon \approx h^2$ (где $h$ --- шаг сетки). Малые $\varepsilon$ дают лучшую точность, но ухудшают обусловленность системы.

\textbf{В.: Почему метод Лагранжа точнее?}

О.: Потому что множитель Лагранжа обеспечивает точное удовлетворение ограничения в оптимальной точке, в то время как штраф только приблизительное.

\textbf{В.: Какой подход лучше всего?}

О.: Нет универсального ответа. Выбор зависит от структуры задачи, требуемой точности, доступного ПО и навыков разработчика.
