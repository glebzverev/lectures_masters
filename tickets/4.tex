\section{Билет 4. Исследование задачи о минимуме функционала энергии. Пространства $H_A$}

\subsection*{Вариационная задача минимизации}

Рассмотрим вариационную задачу для положительно определённого оператора:
\[
J(u) = [u, u]_A - 2(f, u) \to \min, \quad u \in H_A
\]

где $H_A$ --- энергетическое пространство оператора $A$.

Функционал $J(u)$ называется функционалом энергии. Его минимум эквивалентен решению операторного уравнения:
\[
Au = f
\]

в обобщённом смысле.

\subsection*{Критерий минимума}

Функция $u_0 \in H_A$ минимизирует $J(u)$ тогда и только тогда, когда:
\[
\frac{d}{dt} J(u_0 + tv)\bigg|_{t=0} = 0 \quad \forall v \in H_A
\]

Вычисляя:
\[
\frac{d}{dt} [u_0 + tv, u_0 + tv]_A - 2(f, u_0 + tv)\bigg|_{t=0} = 2[u_0, v]_A - 2(f, v) = 0
\]

Таким образом, минимум даёт:
\[
[u_0, v]_A = (f, v) \quad \forall v \in H_A
\]

\subsection*{Структура энергетического пространства $H_A$}

\textbf{Определение:} Энергетическое пространство $H_A$ --- это пополнение $D_A$ по энергетической норме $|\cdot|_A$.

Элемент $u \in H_A$ тогда и только тогда, когда:
\[
\exists \{u_n\} \subset D_A : \quad |u_n - u_m|_A \to 0 \text{ при } n, m \to \infty
\]

Пространство $H_A$ само является гильбертовым пространством со скалярным произведением $[\cdot, \cdot]_A$.

\subsection*{Вложение и отношение между $H$, $H_A$ и $D_A$}

\textbf{Связь норм:}

Для положительно определённого оператора существуют константы $c_1, c_2$ такие, что:
\[
c_1 \|u\|_H \leq |u|_A \leq c_2 \|u\|_H \quad \forall u \in D_A
\]

где $\|u\|_H$ --- норма в исходном гильбертовом пространстве $H$.

Однако в расширенном пространстве $H_A$ норма $|\cdot|_A$ может быть существенно сильнее нормы, индуцированной из $H$.

\textbf{Вложения:}
\[
D_A \subset H_A, \quad H_A \subset H^*
\]

где $H^*$ --- двойственное пространство к $H$.

\subsection*{Теорема о существовании и единственности минимума}

\textbf{Теорема:}

Пусть $A$ --- положительно определённый оператор в $H$, и пусть $f \in H$. Тогда функционал энергии:
\[
J(u) = [u, u]_A - 2(f, u)
\]

имеет единственный минимум на $H_A$, который достигается в точке $u_0$, удовлетворяющей:
\[
[u_0, v]_A = (f, v) \quad \forall v \in H_A
\]

\textit{Доказательство:} 

Функционал $J$ является сильно выпуклым:
\[
J(u) = [u, u]_A - 2(f, u) = [u - u_0, u - u_0]_A - [u_0, u_0]_A + 2(f, u_0) - 2(f, u_0)
\]

Для любого $v \in H_A$:
\[
J(u_0 + v) = J(u_0) + 2[u_0, v]_A - 2(f, v) + [v, v]_A = J(u_0) + [v, v]_A \geq J(u_0)
\]

с равенством только при $v = 0$.

\subsection*{Взаимосвязь между вариационной и операторной формулировками}

Вариационная задача:
\[
J(u) = [u, u]_A - 2(f, u) \to \min
\]

эквивалентна вариационному уравнению:
\[
[u, v]_A = (f, v) \quad \forall v \in H_A
\]

которое в свою очередь эквивалентно операторному уравнению (в обобщённом смысле):
\[
Au = f
\]

\subsection*{Примеры энергетических пространств}

\textbf{Пример 1: задача Дирихле для уравнения Штурма-Лиувилля}

\[
A u = -\frac{d}{dx}(p(x) u') + q(x)u, \quad u(a) = u(b) = 0
\]

Энергетическое пространство:
\[
H_A = \accentset{\circ}{W}_2^1(a, b) = \{u \in W_2^1(a, b) : u(a) = u(b) = 0\}
\]

с нормой $|u|_A = \sqrt{\int_a^b [p(u')^2 + qu^2] dx}$.

\textbf{Пример 2: задача Дирихле для уравнения Лапласа}

\[
Au = -\Delta u, \quad u|_{\partial \Omega} = 0
\]

Энергетическое пространство:
\[
H_A = \accentset{\circ}{W}_2^1(\Omega)
\]

с нормой $|u|_A = \sqrt{\int_\Omega |\nabla u|^2 d\Omega}$.

\subsection*{Дополнительные пояснения}

\textbf{В.: Почему энергетическое пространство называется так?}

О.: Потому что в физических приложениях $[u, u]_A$ часто представляет потенциальную энергию системы. Минимизация функционала энергии соответствует принципу минимума энергии.

\textbf{В.: Как связаны пространства $H$, $H_A$ и $D_A$?}

О.: $D_A$ --- наиболее узкое пространство (содержит только функции, к которым применим оператор в классическом смысле). $H_A$ --- среднее пространство (пополнение $D_A$ по энергетической норме). $H$ --- самое широкое пространство.

\textbf{В.: Почему важно пополнение до $H_A$?}

О.: Потому что в $D_A$ может быть недостаточно «функций», чтобы достичь минимума. Например, для задачи с граничными условиями Дирихле минимум может достигаться на менее гладкой функции, которая лежит в $H_A$, но не в $D_A$.
