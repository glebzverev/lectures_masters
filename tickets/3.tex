\section{Билет 3. Энергетический метод для только положительных операторов}

\subsection*{Определение положительного оператора}

Оператор $A$ в гильбертовом пространстве $H$ называется \textbf{положительным}, если:
\[
(Au, u) \geq 0 \quad \forall u \in D_A, \quad (Au, u) = 0 \Rightarrow u = 0
\]

То есть $A$ положительный, но не обязательно положительно определённый (может не быть эквивалентности норм).

\subsection*{Отличия от положительно определённых операторов}

Для положительно определённого оператора верно $\exists \gamma > 0$:
\[
(Au, u) \geq \gamma^2 \|u\|^2
\]

Для только положительного оператора такой константы может не быть. Энергетическая норма может быть намного «слабее» обычной нормы в $H$.

\subsection*{Энергетическое пространство для положительного оператора}

При работе с положительным оператором также строится энергетическое пространство $H_A$ как пополнение $D_A$ по норме $|\cdot|_A$:
\[
|u|_A = \sqrt{(Au, u)}
\]

Однако из $|u|_A = 0$ следует только $u = 0$, но из $u = 0$ не следует $u = 0$ в исходном пространстве (автоматически верно, но иная структура).

\subsection*{Функционал энергии для положительного оператора}

Для положительного оператора исходная вариационная задача формулируется иначе.

\textbf{Теорема:} 

Пусть $A$ --- положительный оператор, и пусть существует элемент $u_0 \in H_A$ такой, что линейный функционал $l(v) = (f, v)$ продолжим на $H_A$ и ограничен. Тогда существует единственный элемент $\tilde{u} \in H_A$ такой, что:
\[
[\tilde{u}, v]_A = l(v) \quad \forall v \in H_A
\]

Это решение минимизирует функционал (в обобщённом смысле):
\[
F(u) = [u, u]_A - 2l(u) \to \min
\]

\subsection*{Примеры положительных (но не положительно определённых) операторов}

\textbf{Пример 1: задача Неймана}

Рассмотрим оператор Лапласа с условиями Неймана:
\[
Au = -\Delta u, \quad \frac{\partial u}{\partial n}\bigg|_{\partial \Omega} = 0
\]

Имеем:
\[
(Au, u) = \int_\Omega |\nabla u|^2 d\Omega = 0 \text{ для } u = \text{const}
\]

но $\|u\|_{L_2} \neq 0$ для ненулевой константы. Оператор положительный, но не положительно определённый.

\textbf{Пример 2: оператор в бесконечной области}

Для уравнения $Au = -\sum_{i,j} \frac{\partial}{\partial x_i}(a_{ij} \frac{\partial u}{\partial x_j})$ в неограниченной области $\Omega \subset \mathbb{R}^m$:
\[
(Au, u) = \int_\Omega a_{ij} \frac{\partial u}{\partial x_i} \frac{\partial u}{\partial x_j} d\Omega \geq 0
\]

Положительная определённость не гарантирована из-за поведения функций на бесконечности.

\subsection*{Условия разрешимости для положительного оператора}

Для задачи $Au = f$ с положительным оператором разрешимость требует специальных условий.

В частности, для задачи Неймана необходимо условие ортогональности:
\[
\int_\Omega f \, d\Omega = 0
\]

Это отражает факт, что константные функции лежат в ядре оператора.

\subsection*{Обобщённое решение}

Обобщённым решением уравнения $Au = f$ с положительным оператором называется элемент $u_0 \in H_A$ такой, что:
\[
[u_0, v]_A = l(v) \quad \forall v \in H_A
\]

где $l$ --- ограниченный линейный функционал, порождённый $f$.

Такое решение существует и единственно в факторпространстве $H_A / \ker(A)$.

\subsection*{Дополнительные пояснения}

\textbf{В.: Чем энергетический метод для положительного оператора сложнее?}

О.: При работе с только положительным оператором энергетическая норма может обращаться в нуль на ненулевых элементах (на ядре оператора). Нужно работать в факторпространстве по ядру либо налагать дополнительные условия ортогональности.

\textbf{В.: Как практически разрешить задачу с положительным оператором?}

О.: Обычно задачу сводят к положительно определённой, добавляя малый положительный оператор (регуляризация), либо налагают условие нормировки (например, $\int_\Omega u \, d\Omega = 0$ для задачи Неймана).

\textbf{В.: Почему условие $\int_\Omega f \, d\Omega = 0$ необходимо для задачи Неймана?}

О.: Это следует из того, что для любого решения $Au = f$ имеем:
\[
\int_\Omega f \, d\Omega = \int_\Omega Au \, d\Omega = 0 \quad \text{(формула Остроградского)}
\]

Таким образом, условие совместности исходит из самой природы дифференциального оператора.

\textbf{В.: Как задача Неймана связана с задачей Дирихле?}

О.: Задача Дирихле даёт положительно определённый оператор (граничное условие убирает константные функции), а задача Неймана даёт только положительный оператор (константные функции остаются в ядре оператора).
