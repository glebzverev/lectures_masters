\section{Билет 22. Построение проекционной-сеточной схемы для задачи Дирихле для эллиптического уравнения 2 порядка. Анализ сходимости в $L_2$ и $W_2^1$}

\subsection*{Постановка задачи}

На области $\Omega \subset \mathbb{R}^m$ с гладкой границей:
\[
\begin{cases}
-\sum_{i,j=1}^{m} \frac{\partial}{\partial x_i}\left(a_{ij}(x) \frac{\partial u}{\partial x_j}\right) + c(x)u = f(x) & \text{в } \Omega \\
u|_{\partial\Omega} = 0
\end{cases}
\]

где матрица $(a_{ij})$ симметрична и положительно определена.

\subsection*{Вариационная формулировка}

Энергетическое пространство:
\[
H_A = \accentset{\circ}{W}_2^1(\Omega)
\]

Вариационное уравнение:
\[
[u, v]_A = (f, v)_{L_2(\Omega)} \quad \forall v \in H_A
\]

где:
\[
[u, v]_A = \int_\Omega \left[\sum_{i,j} a_{ij} \frac{\partial u}{\partial x_i} \frac{\partial v}{\partial x_j} + c u v\right] d\Omega
\]

\subsection*{Триангуляция области}

Покрываем область $\Omega$ триангуляцией $\mathcal{T}_h$:
\[
\Omega = \bigcup_{T \in \mathcal{T}_h} T
\]

где каждый элемент $T$ --- симплекс (треугольник в $\mathbb{R}^2$, тетраэдр в $\mathbb{R}^3$).

Параметр триангуляции: $h = \max_{T} \text{diam}(T)$.

\subsection*{Пространство конечных элементов}

На каждом треугольнике $T$ используем линейные функции:
\[
v|_T(x, y) = a_T + b_T x + c_T y
\]

Глобальное пространство:
\[
V_h = \{v_h \in C(\overline{\Omega}) : v_h|_T \text{ --- линейна для всех } T \in \mathcal{T}_h, \, v_h|_{\partial\Omega} = 0\}
\]

Размерность: количество внутренних узлов триангуляции.

\subsection*{Базисные функции}

Для каждого внутреннего узла $x_i$ определяем базисную функцию $\varphi_i$:
\[
\varphi_i(x_j) = \delta_{ij}, \quad \text{на каждом элементе } \varphi_i \text{ линейна}
\]

Носитель: объединение всех треугольников, содержащих узел $x_i$.

\subsection*{Дискретная схема}

Приближённое решение:
\[
u_h = \sum_{i=1}^{N_h} u_i \varphi_i(x)
\]

где $N_h$ --- количество внутренних узлов.

Система уравнений:
\[
\sum_{j=1}^{N_h} A_{ij} u_j = b_i, \quad i = 1, \ldots, N_h
\]

где:
\[
A_{ij} = [\varphi_j, \varphi_i]_A, \quad b_i = (f, \varphi_i)_{L_2}
\]

\subsection*{Вычисление матричных элементов}

Матрица $A$ собирается суммированием вкладов от каждого элемента:
\[
A_{ij} = \sum_{T \in \mathcal{T}_h} A_{ij}^T
\]

где локальный вклад:
\[
A_{ij}^T = \int_T \left[\sum_{k,l} a_{kl} \frac{\partial \varphi_i}{\partial x_k} \frac{\partial \varphi_j}{\partial x_l} + c \varphi_i \varphi_j\right] dT
\]

На каждом элементе градиенты $\nabla \varphi_i$ и $\nabla \varphi_j$ постоянны.

\subsection*{Свойства матрицы системы}

1. \textbf{Симметрия:} $A_{ij} = A_{ji}$

2. \textbf{Положительная определённость:} следует из положительной определённости оператора

3. \textbf{Разреженность:} $A_{ij} \neq 0$ только если $\varphi_i$ и $\varphi_j$ имеют пересекающиеся носители

4. \textbf{Масштабирование:} $\|A\| \sim O(1/h^2)$

\subsection*{Теорема о сходимости в $W_2^1$}

\textbf{Теорема:}

Пусть триангуляция регулярна (все углы ограничены снизу положительной константой), и пусть $u \in W_2^2(\Omega)$. Тогда:
\[
\boxed{\|u - u_h\|_{W_2^1(\Omega)} \leq C h \|u\|_{W_2^2(\Omega)} \leq C h \|f\|_{L_2(\Omega)}}
\]

\textit{Идея доказательства:}

1. Оптимальность приближения в энергии: $|u - u_h|_A \leq \inf_{v_h \in V_h} |u - v_h|_A$

2. Интерполяция: берём интерполянта $I_h u \in V_h$ и оцениваем $|u - I_h u|_A \leq C h \|u\|_{W_2^2}$

3. Эквивалентность норм: $|u|_A \approx \|u\|_{W_2^1}$

\subsection*{Теорема о сходимости в $L_2$}

\textbf{Теорема (Nitsche trick):}

При регулярности оператора ($u \in W_2^2$ для $f \in L_2$):
\[
\boxed{\|u - u_h\|_{L_2(\Omega)} \leq C h^2 \|f\|_{L_2(\Omega)}}
\]

\textit{Идея:} Вводим вспомогательную задачу с правой частью $(u - u_h)$, используем регулярность и ортогональность ошибки в энергии.

\subsection*{Практические рекомендации}

- \textbf{Качество сетки:} избегать острых углов в треугольниках (минимальный угол $> 20°$)

- \textbf{Адаптивная сетка:} сгущать сетку в областях с большой ошибкой (оцениваемой через локальные невязки)

- \textbf{Система уравнений:} для больших систем использовать итеративные методы с предобусловливанием (PCG, GMRES)

\subsection*{Многосеточные методы}

Для эффективного решения системы можно использовать многосеточные методы, которые снижают сложность с $O(N^2)$ (итеративные методы) до $O(N)$.

\subsection*{Дополнительные пояснения}

\textbf{В.: Почему требуется регулярность триангуляции?}

О.: Острые углы приводят к плохой аппроксимационной способности линейных функций и плохой обусловленности матрицы системы.

\textbf{В.: Как практически вычислить интегралы $A_{ij}$?}

О.: Обычно аналитически на каждом элементе (градиенты линейных функций постоянны), либо численно с помощью квадратурных формул.

\textbf{В.: Каковы требования к размеру сетки?}

О.: Из оценки $\|u - u_h\|_{L_2} \leq Ch^2 \|f\|$. Для получения точности $\varepsilon$ нужно $h \sim \sqrt{\varepsilon}$. На практике используют апостериорные оценки ошибки для адаптивного уточнения.

\textbf{В.: Как эта теория распространяется на неструктурированные сетки?}

О.: Основные результаты остаются верны, но константы в оценках могут зависеть от параметра регулярности триангуляции (отношения наибольшего размера к наименьшему).
