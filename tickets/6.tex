\section{Билет 6. Процесс Ритца построения минимизирующей последовательности энергетического функционала}

\subsection*{Основная идея метода Ритца}

Метод Ритца состоит в замене исходной вариационной задачи на бесконечномерном пространстве $H_A$ приблизительной задачей на конечномерном подпространстве.

Вместо поиска $u \in H_A$ такого, что $[u, v]_A = (f, v)$ для всех $v \in H_A$, ищем $u_n \in H_n$, где $H_n$ --- конечномерное подпространство $H_A$.

\subsection*{Выбор базисных функций}

Выбираем полную в $H_A$ последовательность линейно независимых функций $\{\varphi_1, \varphi_2, \ldots\}$:
\begin{enumerate}
\item каждая $\varphi_k$ удовлетворяет однородным граничным условиям;
\item $\varphi_k \in H_A$ (или $\varphi_k \in D_A$);
\item система полна: линейные комбинации $\{\varphi_1, \ldots, \varphi_n\}$ всюду плотны в $H_A$ при $n \to \infty$.
\end{enumerate}

\subsection*{Конечномерное подпространство}

Определяем подпространство размерности $n$:
\[
H_n = \text{span}\{\varphi_1, \varphi_2, \ldots, \varphi_n\}
\]

Приближённое решение ищем в виде:
\[
u_n(x) = \sum_{k=1}^{n} a_k \varphi_k(x)
\]

\subsection*{Вариационная формулировка на подпространстве}

На подпространстве $H_n$ ищем $u_n$ такой, что:
\[
[u_n, \varphi_j]_A = (f, \varphi_j) \quad \forall j = 1, \ldots, n
\]

Подставляя $u_n = \sum_{k=1}^{n} a_k \varphi_k$:
\[
\sum_{k=1}^{n} a_k [\varphi_k, \varphi_j]_A = (f, \varphi_j), \quad j = 1, \ldots, n
\]

\subsection*{Система линейных уравнений}

Получаем систему алгебраических уравнений:
\[
\boxed{\sum_{k=1}^{n} A_{jk} a_k = b_j, \quad j = 1, \ldots, n}
\]

где матрица Грама энергетического скалярного произведения:
\[
A_{jk} = [\varphi_k, \varphi_j]_A = (A\varphi_k, \varphi_j)
\]

и правая часть:
\[
b_j = (f, \varphi_j)
\]

\subsection*{Свойства матрицы системы}

Матрица $\widehat{A} = (A_{jk})$ обладает следующими свойствами:

1. \textbf{Симметричность:} $A_{jk} = A_{kj}$ (следует из симметричности оператора $A$)

2. \textbf{Положительная определённость:} Матрица положительно определена, так как
\[
\sum_{j,k} A_{jk} c_j c_k = \left[\sum_{j} c_j \varphi_j, \sum_k c_k \varphi_k\right]_A = \left|\sum_j c_j \varphi_j\right|_A^2 > 0
\]
для любого ненулевого вектора $(c_1, \ldots, c_n)$.

3. \textbf{Хорошая обусловленность:} При выборе ортогональных (в энергетической норме) базисных функций матрица становится диагональной.

\subsection*{Построение минимизирующей последовательности}

Для последовательности $n = 1, 2, 3, \ldots$:
\begin{enumerate}
\item Выбираем первые $n$ базисных функций $\varphi_1, \ldots, \varphi_n$

\item Составляем систему размера $n \times n$

\item Решаем систему и находим коэффициенты $a_1^{(n)}, \ldots, a_n^{(n)}$

\item Строим приближённое решение $u_n = \sum_{k=1}^{n} a_k^{(n)} \varphi_k$
\end{enumerate}

Таким образом получается последовательность приближённых решений $\{u_n\}$.

\subsection*{Примеры базисных функций}

\textbf{Пример 1: степенные функции}

Для задачи на отрезке $[0, L]$ с нулевыми условиями Дирихле:
\[
\varphi_k(x) = x^k(L - x)^m, \quad k = 1, 2, \ldots
\]

Такие функции удовлетворяют граничным условиям и составляют полную систему.

\textbf{Пример 2: тригонометрические функции}

Для периодических условий:
\[
\varphi_k(x) = \sin(k\pi x / L), \quad k = 1, 2, \ldots
\]

Эти функции ортогональны в энергетической норме для некоторых операторов.

\textbf{Пример 3: кусочно-линейные функции}

На сетке с узлами $x_0 < x_1 < \cdots < x_n$:
\[
\varphi_i(x) = \begin{cases}
\frac{x - x_{i-1}}{h_i} & x \in (x_{i-1}, x_i) \\
\frac{x_{i+1} - x}{h_{i+1}} & x \in (x_i, x_{i+1}) \\
0 & \text{иначе}
\end{cases}
\]

\subsection*{Дополнительные пояснения}

\textbf{В.: Почему матрица системы положительно определена?}

О.: Потому что $(A\varphi, \varphi) \geq \gamma^2 \|\varphi\|^2 > 0$ для любого ненулевого $\varphi$ (положительная определённость оператора). Это гарантирует положительную определённость матрицы Грама.

\textbf{В.: Как выбрать базисные функции?}

О.: Нужно выбрать так, чтобы: (1) удовлетворяли граничным условиям; (2) были линейно независимы; (3) их линейные комбинации хорошо аппроксимировали точное решение. На практике часто используют полиномы или кусочно-полиномиальные функции.

\textbf{В.: Почему нужна полнота системы?}

О.: Полнота гарантирует, что при $n \to \infty$ подпространства $H_n$ исчерпывают всё пространство $H_A$. Это обеспечивает сходимость метода Ритца.
