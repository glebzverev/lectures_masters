\section{Билет 8. Построение обобщённого решения краевой задачи Дирихле для уравнения Штурма-Лиувилля}

\subsection*{Постановка задачи Штурма-Лиувилля}

Рассмотрим краевую задачу на отрезке $[a, b]$:
\[
\boxed{\begin{cases}
-\frac{d}{dx}\left(p(x) \frac{du}{dx}\right) + q(x)u = f(x), & x \in (a, b) \\
u(a) = 0, \quad u(b) = 0
\end{cases}}
\]

где:
\begin{itemize}
\item $p(x) \geq p_0 > 0$ --- коэффициент при производной
\item $q(x) \geq 0$ --- потенциал
\item $f \in L_2(a, b)$ --- правая часть
\end{itemize}

\subsection*{Вариационная формулировка}

Умножим уравнение на пробную функцию $v \in \accentset{\circ}{W}_2^1(a,b)$ (со своторыми нулевыми на границе) и проинтегрируем:

\[
-\int_a^b v \frac{d}{dx}\left(p(x) \frac{du}{dx}\right) dx + \int_a^b q(x)u v \, dx = \int_a^b f v \, dx
\]

Применяя интегрирование по частям к первому слагаемому:
\[
\int_a^b p(x) \frac{du}{dx} \frac{dv}{dx} dx + \int_a^b q(x) u v \, dx = \int_a^b f v \, dx
\]

(граничные члены исчезают благодаря нулевым условиям на $v$).

\subsection*{Энергетическое пространство и скалярное произведение}

Определим энергетическое скалярное произведение:
\[
[u, v]_A := \int_a^b \left[p(x) u'(x) v'(x) + q(x) u(x) v(x)\right] dx
\]

Энергетическая норма:
\[
|u|_A := \sqrt{[u, u]_A}
\]

Энергетическое пространство:
\[
H_A = \accentset{\circ}{W}_2^1(a, b) := \{u \in W_2^1(a,b) : u(a) = u(b) = 0\}
\]

\subsection*{Обобщённая формулировка}

\textbf{Определение:} Обобщённым решением задачи Дирихле называется функция $u \in H_A$ такая, что:
\[
\boxed{[u, v]_A = (f, v)_{L_2} \quad \forall v \in H_A}
\]

где $(f, v)_{L_2} = \int_a^b f(x) v(x) dx$.

\subsection*{Положительная определённость оператора}

\textbf{Теорема:}

Оператор Штурма-Лиувилля с краевыми условиями Дирихле положительно определён.

\textit{Доказательство:}

\[
[u, u]_A = \int_a^b p(x) (u')^2 dx + \int_a^b q(x) u^2 dx \geq p_0 \int_a^b (u')^2 dx
\]

По неравенству Фридрихса для функций с нулевыми граничными условиями:
\[
\|u\|_{L_2}^2 \leq C \|u'\|_{L_2}^2
\]

Объединяя:
\[
[u, u]_A \geq p_0 \|u'\|_{L_2}^2 \geq \frac{p_0}{C} \|u\|_{L_2}^2 =: \gamma^2 \|u\|_{L_2}^2
\]

\subsection*{Существование и единственность решения}

\textbf{Теорема:}

Для любой правой части $f \in L_2(a, b)$ существует единственное обобщённое решение $u \in H_A$ краевой задачи Дирихле для уравнения Штурма-Лиувилля.

\textit{Доказательство:}

По теореме Рисса об ограниченных функционалах в гильбертовом пространстве, функционал $(f, v)$ ограничен в $H_A$:
\[
|(f, v)| \leq \|f\|_{L_2} \|v\|_{L_2} \leq \|f\|_{L_2} C |v|_A
\]

Таким образом, существует единственный $u \in H_A$ такой, что:
\[
[u, v]_A = (f, v) \quad \forall v \in H_A
\]

\subsection*{Связь с классическим решением}

Если обобщённое решение $u \in H_A$ достаточно гладкое, скажем $u \in C^2[a, b]$, то оно удовлетворяет исходному дифференциальному уравнению в классическом смысле.

\textbf{Замечание:} Обобщённое решение может быть менее гладким, чем $C^2$, но всё равно содержит физически значимую информацию.

\subsection*{Эквивалентная минимизационная задача}

Обобщённое решение равносильно минимизации функционала энергии:
\[
J(u) = [u, u]_A - 2(f, u)_{L_2} \to \min
\]

\subsection*{Дополнительные пояснения}

\textbf{В.: Что такое обобщённое решение и когда оно нужно?}

О.: Обобщённое решение определяется через интегральное тождество, а не через дифференциальное уравнение. Оно нужно, когда классическое решение не существует или недостаточно гладкое.

\textbf{В.: Почему граничные условия автоматически учитываются?}

О.: Потому что пространство $H_A$ содержит только функции, обращающиеся в нуль на границе. Таким образом, условия Дирихле встроены в само определение пространства.

\textbf{В.: Как проверить, что решение достаточно гладко?}

О.: Если $f \in W_2^k$, то по теории регулярности эллиптических уравнений $u \in W_2^{k+2}$. При $k = 0$ (т.е. $f \in L_2$) получаем $u \in W_2^2$.
