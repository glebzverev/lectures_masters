\section{Билет 1. Метод Ритца на примере задачи об упругой пластине}


\subsection*{Формулировка задачи}

Рассматривается упругая тонкая пластина, занимающая область
\[
\Omega \subset \mathbb{R}^2, \qquad S = \partial \Omega.
\]

Прогиб пластины $w(x,y)$ под действием распределённой нагрузки $q(x,y)$
удовлетворяет уравнению Софи Жермен:
\[
\Delta^2 w
=
\frac{\partial^4 w}{\partial x^4}
+ 2 \frac{\partial^4 w}{\partial x^2 \partial y^2}
+ \frac{\partial^4 w}{\partial y^4}
=
\frac{q(x,y)}{D},
\]
где $D$ — жёсткость пластины.

Для защемлённого края выполняются краевые условия:
\[
w = 0, \qquad
\frac{\partial w}{\partial n} = 0
\quad \text{на } S.
\]

\begin{figure}[h!]
	\begin{center}
		\includegraphics[scale=0.3]{images/1_0.png}
	\end{center}
	\label{12}
	\caption{Визуальизация задачи об упругой пластине}
\end{figure}

\subsection*{Вариационная постановка}

Задача эквивалентна задаче минимизации функционала полной энергии:
\[
\boxed{
J(w)
=
\iint\limits_{\Omega}
\left[
\frac{1}{2} (\Delta w)^2 - f(x,y) w
\right]
\, d\Omega
\;\longrightarrow\;
\inf
}
\]
в классе допустимых функций
\[
M
=
\left\{
w \in H^2(\Omega) :
w|_{S}=0,\;
\frac{\partial w}{\partial n}\big|_{S}=0
\right\}.
\]

Минимум функционала соответствует физически реализуемому прогибу пластины.

\subsection*{Метод Ритца}

Идея метода Ритца состоит в том, чтобы искать приближённое решение
в виде конечной линейной комбинации:
\[
w_n(x,y)
=
\sum_{k=1}^{n} a_k \psi_k(x,y),
\]
где координатные функции $\psi_k$ выбираются так, чтобы:
\begin{enumerate}
\item каждая $\psi_k$ удовлетворяла краевым условиям;
\item $\psi_k \in C^3(\overline{\Omega})$;
\item система $\{\psi_k\}$ была полной в $M$.
\end{enumerate}

Подстановка $w_n$ в функционал приводит к функции
\[
J(w_n) = J_n(a_1,\dots,a_n),
\]
минимум которой находится из системы уравнений:
\[
\boxed{
\sum_{k=1}^{n} A_{ik} a_k = B_i,
\qquad i = 1,\dots,n,
}
\]
где
\[
A_{ik}
=
\iint\limits_{\Omega}
\Delta \psi_i \, \Delta \psi_k \, d\Omega,
\qquad
B_i
=
\iint\limits_{\Omega}
f \, \psi_i \, d\Omega.
\]

Полученная система линейна и имеет единственное решение.

\subsection*{Сходимость метода}

Минимальные значения функционала удовлетворяют цепочке:
\[
J_1^{(0)} \ge J_2^{(0)} \ge \dots \ge \inf J(w).
\]
При выполнении условий полноты системы $\{\psi_k\}$
приближённые решения $w_n$ сходятся к точному решению задачи.

\newpage
\subsection*{Дополнительные вопросы и пояснения}

\paragraph{1. Почему уравнение пластины имеет четвёртый порядок?}

Поскольку изгиб пластины определяется кривизнами,
которые зависят от вторых производных прогиба.
Уравнения равновесия связывают моменты и поперечные силы,
что приводит к оператору четвёртого порядка $\Delta^2$.

\paragraph{2. Почему задача формулируется вариационно?}

Потому что упругая система в состоянии равновесия
минимизирует потенциальную энергию.
Вариационная постановка:
\[
\delta J(w) = 0
\]
эквивалентна уравнению Софи Жермен
и позволяет применять приближённые методы.

\paragraph{3. В каком функциональном пространстве ищется решение?}

Решение ищется в пространстве $H^2(\Omega)$,
поскольку в функционале участвуют вторые производные $w$.
Краевые условия задают подпространство допустимых функций.

\paragraph{4. Как выбираются координатные функции $\psi_k$?}

Координатные функции выбираются так,
чтобы они автоматически удовлетворяли краевым условиям.
Например, для прямоугольной области:
\[
\psi_{mn}(x,y)
=
x^2(a-x)^2 y^2(b-y)^2 P_m(x) P_n(y),
\]
где $P_m, P_n$ — полиномы.

\paragraph{5. Почему система уравнений метода Ритца линейна?}

Потому что функционал $J(w)$ квадратичен по $w$,
а приближение $w_n$ линейно зависит от коэффициентов $a_k$.
Минимизация приводит к линейной системе.

\paragraph{6. В чём физический смысл метода Ритца?}

Метод Ритца ищет форму прогиба пластины,
которая минимизирует энергию изгиба
в ограниченном классе допустимых форм.
При увеличении числа координатных функций
форма всё точнее приближается к реальной.

\paragraph{7. Когда метод Ритца применять нельзя напрямую?}

Если функционал не ограничен снизу
или минимум не достигается (контрпример Вейерштрасса),
необходимо уточнение класса функций
или использование других вариационных методов.