\section{Билет 20. Билинейные сплайны в плоской области и их применение к построению приближённого решения задачи Дирихле для уравнения Лапласа}

\subsection*{Определение билинейного сплайна}

На прямоугольной области $\Omega = [0, a] \times [0, b]$ введём равномерную сетку:
\[
0 = x_0 < x_1 < \cdots < x_{N_x} = a, \quad h_x = a / N_x
\]
\[
0 = y_0 < y_1 < \cdots < y_{N_y} = b, \quad h_y = b / N_y
\]

\subsection*{Базисные функции}

На каждом прямоугольнике $[x_i, x_{i+1}] \times [y_j, y_{j+1}]$ функция билинейна:
\[
Q_{ij}(x, y) = \varphi_i(x) \psi_j(y)
\]

где $\varphi_i$ и $\psi_j$ --- кусочно-линейные функции одной переменной:
\[
\varphi_i(x) = \begin{cases}
\frac{x - x_{i-1}}{h_x} & x \in [x_{i-1}, x_i] \\
\frac{x_{i+1} - x}{h_x} & x \in [x_i, x_{i+1}] \\
0 & \text{иначе}
\end{cases}
\]

и аналогично для $\psi_j(y)$.

\subsection*{Базис пространства билинейных сплайнов}

Полное семейство базисных функций:
\[
Q_{ij}(x, y) = \varphi_i(x) \psi_j(y), \quad i = 0, \ldots, N_x, \quad j = 0, \ldots, N_y
\]

Пространство:
\[
\mathcal{S}_{1,1}^{(2)} = \text{span}\{Q_{ij} : 0 \leq i \leq N_x, 0 \leq j \leq N_y\}
\]

Размерность: $(N_x + 1)(N_y + 1)$.

\subsection*{Свойства базисных функций}

1. \textbf{Интерполяция:} $Q_{ij}(x_k, y_l) = \delta_{ik} \delta_{jl}$

2. \textbf{Локальный носитель:} $\text{supp} Q_{ij} = [x_{i-1}, x_{i+1}] \times [y_{j-1}, y_{j+1}]$ (максимум четыре соседних прямоугольника)

3. \textbf{Непрерывность:} функции непрерывны во всей области

4. \textbf{Частные производные:}
\[
\frac{\partial Q_{ij}}{\partial x} = \varphi_i'(x) \psi_j(y), \quad \frac{\partial Q_{ij}}{\partial y} = \varphi_i(x) \psi_j'(y)
\]

(кусочно-постоянные в каждой переменной)

\subsection*{Энергетическое скалярное произведение}

Для задачи Лапласа с условиями Дирихле:
\[
[u, v]_A = \int_\Omega \nabla u \cdot \nabla v \, d\Omega = \int_\Omega \left(\frac{\partial u}{\partial x} \frac{\partial v}{\partial x} + \frac{\partial u}{\partial y} \frac{\partial v}{\partial y}\right) dx dy
\]

\subsection*{Построение приближённого решения}

\textbf{Шаг 1:} Дискретизуем граничные условия. Пусть $u|_{\partial\Omega} = g$. Берём узлы на границе и полагаем:
\[
a_{ij} = g(x_i, y_j) \quad \text{для граничных узлов}
\]

\textbf{Шаг 2:} Для внутренних узлов выписываем вариационное уравнение:
\[
[u_h, Q_{kl}]_A = 0 \quad \text{для внутренних } (k, l)
\]

(так как $f = 0$ для уравнения Лапласа).

\textbf{Шаг 3:} Подставляя $u_h = \sum_{i,j} a_{ij} Q_{ij}$, получаем систему линейных уравнений для неизвестных $a_{ij}$ внутри области.

\subsection*{Явный вид системы уравнений}

Для равномерной сетки и уравнения Лапласа, интегрируя по частям и учитывая граничные условия, получаем:
\[
a_{k+1,l} + a_{k-1,l} + a_{k,l+1} + a_{k,l-1} - 4 a_{k,l} = 0
\]

(стандартное пятиточечное разностное соотношение).

Это соотношение верно для внутренних узлов; на границе полагаем известные значения.

\subsection*{Теоремы аппроксимации}

\textbf{Теорема 1:}

Для функции $u \in C^2(\overline{\Omega})$ существует билинейная сплайн-функция $u_h \in \mathcal{S}_{1,1}^{(2)}$ такая, что:
\[
\|u - u_h\|_{L_2(\Omega)} \leq C (h_x^2 + h_y^2) \|u\|_{C^2(\overline{\Omega})}
\]
\[
\|u - u_h\|_{W_2^1(\Omega)} \leq C \max(h_x, h_y) \|u\|_{C^2(\overline{\Omega})}
\]

\textbf{Теорема 2 (сходимость для задачи Дирихле):}

Если $u \in W_2^2(\Omega)$ --- решение задачи Лапласа, то приближённое решение $u_h$ удовлетворяет:
\[
\|u - u_h\|_{W_2^1(\Omega)} \leq C h \|u\|_{W_2^2(\Omega)}
\]

где $h = \max(h_x, h_y)$.

\subsection*{Матричная структура системы}

Матрица системы для внутренних узлов имеет блочную трёхдиагональную структуру (для упорядочивания узлов по строкам сетки):
\[
A = \begin{pmatrix}
T & -I & 0 & \cdots \\
-I & T & -I & \cdots \\
0 & -I & T & \cdots \\
\vdots & \vdots & \vdots & \ddots
\end{pmatrix}
\]

где $T$ --- трёхдиагональная матрица размера $N_x \times N_x$, $I$ --- единичная матрица.

\subsection*{Пример: прямоугольная область}

На единичном квадрате $[0, 1]^2$ с равномерной сеткой $N \times N$ и условиями Дирихле $u|_{\partial\Omega} = 0$:

1. Количество внутренних узлов: $(N-1)^2$

2. Размер системы: $(N-1)^2 \times (N-1)^2$

3. Число ненулевых элементов в матрице: $O(N^2)$ (матрица разреженная)

4. Для решения можно использовать методы для разреженных систем (ILU, CG, GMRES и т.д.)

\subsection*{Дополнительные пояснения}

\textbf{В.: Почему используются именно билинейные функции?}

О.: Потому что они являются минимально необходимыми для лучшей аппроксимации (степень 1 в каждой переменной). Более высокие степени усложняют вычисления без значительного улучшения.

\textbf{В.: Как изменяется система при неравномерной сетке?}

О.: Коэффициенты становятся разными, но структура остаётся блочной трёхдиагональной. Вычисления усложняются, но принцип не меняется.

\textbf{В.: Какова сложность решения системы?}

О.: Для $(N-1)^2$ неизвестных полная гауссова элиминация даёт $O(N^6)$ операций. Специализированные методы (блочные прогонки, многосеточные методы) снижают это до $O(N^3)$ или даже $O(N^2 \log N)$.

\textbf{В.: Как обобщить на более сложные области?}

О.: Используется триангуляция области и линейные базисные функции на треугольниках (вместо билинейных на прямоугольниках). Это метод конечных элементов на неструктурированной сетке.
