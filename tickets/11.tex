\section{Билет 11. Краевая задача для ОДУ чётного порядка. Краевая задача для системы ОДУ}

\subsection*{Задача для ОДУ чётного порядка}

Рассмотрим краевую задачу для дифференциального уравнения 4-го порядка:
\[
\boxed{\begin{cases}
\frac{d^4u}{dx^4} = f(x), & x \in (a, b) \\
u(a) = u(b) = 0, & \frac{d^2u}{dx^2}(a) = \frac{d^2u}{dx^2}(b) = 0
\end{cases}}
\]

Это типичная задача о прогибе балки под нагрузкой (защемлённые края).

\subsection*{Вариационная формулировка для задачи 4-го порядка}

Умножим на пробную функцию $v$ (с соответствующими нулевыми граничными условиями) и проинтегрируем дважды по частям:
\[
\int_a^b \frac{d^2u}{dx^2} \frac{d^2v}{dx^2} dx = \int_a^b f(x) v(x) dx
\]

Энергетическое скалярное произведение:
\[
[u, v]_A = \int_a^b \frac{d^2u}{dx^2} \frac{d^2v}{dx^2} dx
\]

Энергетическое пространство:
\[
H_A = \{u \in W_2^2(a, b) : u(a) = u(b) = 0, \, u''(a) = u''(b) = 0\}
\]

или в обозначении со стёрикулюсом:
\[
H_A = \accentset{\circ}{W}_2^2(a, b)
\]

\subsection*{Система ОДУ первого порядка}

Рассмотрим систему:
\[
\boxed{\begin{cases}
\frac{du}{dx} = A(x) u + f(x), & x \in (a, b) \\
u(a) = u_0 \quad \text{(начальное условие)}
\end{cases}}
\]

где $u(x) \in \mathbb{R}^n$ --- вектор неизвестных функций, $A(x)$ --- матрица коэффициентов.

Это система с начальными условиями, а не краевая задача. Решение существует по теореме Пикара.

\subsection*{Система ОДУ второго порядка с краевыми условиями}

Более типична система краевых задач:
\[
\boxed{\begin{cases}
-\frac{d}{dx}\left(P(x) \frac{du}{dx}\right) + Q(x)u = f(x), & x \in (a, b) \\
u(a) = u(b) = 0
\end{cases}}
\]

где:
- $u(x) \in \mathbb{R}^n$ --- вектор неизвестных
- $P(x)$ --- матрица размера $n \times n$ (положительно определена)
- $Q(x)$ --- матрица размера $n \times n$ (положительно полуопределена)
- $f(x) \in \mathbb{R}^n$ --- вектор правых частей

\subsection*{Энергетическая пространство для систем}

Энергетическое скалярное произведение в пространстве вектор-функций:
\[
[u, v]_A = \int_a^b \left[\left(\frac{du}{dx}\right)^T P(x) \frac{dv}{dx} + u^T(x) Q(x) v(x)\right] dx
\]

Энергетическое пространство:
\[
H_A = \{u \in W_2^1(a, b; \mathbb{R}^n) : u(a) = u(b) = 0\}
\]

\subsection*{Вариационная формулировка для системы}

Обобщённым решением системы называется $u \in H_A$ такой, что:
\[
[u, v]_A = \int_a^b f^T(x) v(x) dx \quad \forall v \in H_A
\]

\subsection*{Положительная определённость}

Система положительно определена, если:
\[
[u, u]_A = \int_a^b \left[\left(\frac{du}{dx}\right)^T P(x) \frac{du}{dx} + u^T Q(x) u\right] dx \geq \gamma \|u\|_{L_2}^2
\]

что гарантируется, если:
- $P(x) \geq p_0 I$ (равномерно положительно определена)
- $Q(x) \geq 0$ (неотрицательна)

\subsection*{Метод Ритца для систем}

Приближённое решение:
\[
u_n(x) = \sum_{k=1}^{n} a_k \varphi_k(x)
\]

где $\varphi_k$ --- матрицы-функции со значениями в $\mathbb{R}^m$ (где $m$ --- размерность системы).

Система уравнений:
\[
\sum_{k=1}^{n} A_{jk} a_k = b_j, \quad j = 1, \ldots, n
\]

где коэффициенты матрицы:
\[
A_{jk} = [\varphi_k, \varphi_j]_A, \quad b_j = (\varphi_j, f)_{L_2}
\]

\subsection*{Пример: система для эластичности}

Уравнения плоской теории упругости:
\[
-\frac{\partial}{\partial x}\left(\mu \frac{\partial u}{dx}\right) - \frac{\partial}{\partial y}\left(\mu \frac{\partial u}{dy}\right) - \frac{\partial}{\partial x}\left(\lambda + 2\mu) \frac{\partial v}{dx}\right) - \ldots = f_x
\]

где $u, v$ --- компоненты вектора смещения, $\lambda, \mu$ --- параметры Ламе.

\subsection*{Дополнительные пояснения}

\textbf{В.: Почему для уравнений 4-го порядка нужны два граничных условия на каждом конце?}

О.: Потому что уравнение содержит производные до 4-го порядка. По теории ОДУ, для единственности решения нужно 4 условия, которые естественно разделяются (по 2 на каждом конце).

\textbf{В.: Как система ОДУ отличается от одного ОДУ?}

О.: Основное отличие в размерности неизвестных. Теория вариационных методов распространяется на системы практически без изменений благодаря линейности операторов.

\textbf{В.: Зачем нужна матрица $P(x)$?}

О.: Она описывает анизотропию материала. Если материал анизотропный, коэффициенты при разных направлениях разные, что отражается в матричной форме оператора.

\textbf{В.: Может ли система быть несимметричной?}

О.: Да, в общем случае. Но для применения энергетического метода и метода Ритца нужна симметричность, иначе не имеет смысла энергетическая норма.
