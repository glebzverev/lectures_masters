\section{Билет 18. Степенные сплайны дефекта 1. Основные теоремы аппроксимации сплайнами первой степени}

\subsection*{Кусочно-линейные сплайны}

На сетке $a = x_0 < x_1 < \cdots < x_N = b$ определим кусочно-линейные функции:
\[
\varphi_i(x) = \begin{cases}
\frac{x - x_{i-1}}{h_i} & x \in [x_{i-1}, x_i] \\
\frac{x_{i+1} - x}{h_{i+1}} & x \in [x_i, x_{i+1}] \\
0 & x \notin [x_{i-1}, x_{i+1}]
\end{cases}
\]

где $h_i = x_i - x_{i-1}$.

Пространство кусочно-линейных сплайнов дефекта 1 (непрерывных):
\[
\mathcal{S}_{1,1} = \text{span}\{\varphi_0, \varphi_1, \ldots, \varphi_N\}
\]

размерности $N + 1$.

\subsection*{Базовые свойства}

1. \textbf{Базис интерполяции:} $\varphi_i(x_j) = \delta_{ij}$

2. \textbf{Носитель:} $\text{supp} \varphi_i = [x_{i-1}, x_{i+1}]$ (содержит максимум два подотрезка)

3. \textbf{Полнота:} система $\{\varphi_k\}$ полна в $C[a, b]$

4. \textbf{Ортогональность в энергии:} Для оператора с диагональными коэффициентами функции почти ортогональны в энергетической норме.

\subsection*{Главная теорема аппроксимации (Теорема 1)}

\textbf{Теорема (аппроксимация в норме $W_2^1$):}

Для функции $u \in W_2^2(a, b)$ существует кусочно-линейная функция $s_N(x) \in \mathcal{S}_{1,1}$ такая, что:
\[
\|u - s_N\|_{L_2(a,b)} \leq C_1 h^2 \|u''\|_{L_2(a,b)}
\]
\[
\|u' - s_N'\|_{L_2(a,b)} \leq C_2 h \|u''\|_{L_2(a,b)}
\]

где константы $C_1, C_2$ не зависят от $h$ и $u$, и $h = \max_i h_i$.

\textit{Доказательство (идея):} На каждом подотрезке $[x_{i-1}, x_i]$ функция $u$ аппроксимируется линейной интерполяцией $L_i(x)$. Локальная оценка погрешности:
\[
|u(x) - L_i(x)| \leq \frac{h_i^2}{8} \max_{\xi \in [x_{i-1}, x_i]} |u''(\xi)|
\]

Суммирование по всем подотрезкам и применение неравенства Коши-Буняковского дают глобальную оценку.

\subsection*{Теорема 2: аппроксимация в $C$-норме}

\textbf{Теорема:}

Для функции $u \in C^2[a, b]$ существует $s_N \in \mathcal{S}_{1,1}$ такая, что:
\[
\|u - s_N\|_{C(a,b)} \leq C h^2 \|u''\|_{C(a,b)}
\]
\[
\|u' - s_N'\|_{C(a,b)} \leq C h \|u''\|_{C(a,b)}
\]

\subsection*{Оптимальный выбор интерполянта}

Наиболее естественный выбор --- интерполяция по узлам:
\[
s_N(x) = \sum_{i=0}^{N} u(x_i) \varphi_i(x)
\]

Этот выбор даёт оптимальные константы в оценках и максимальную точность.

\subsection*{Энергетическое скалярное произведение}

Для задачи Штурма-Лиувилля энергетическое скалярное произведение между базисными функциями:
\[
[\varphi_i, \varphi_j]_A = \int_a^b \left[p(x) \varphi_i'(x) \varphi_j'(x) + q(x) \varphi_i(x) \varphi_j(x)\right] dx
\]

Матрица $A_{ij} = [\varphi_i, \varphi_j]_A$ является трёхдиагональной (так как носители функций пересекаются только в соседних узлах).

\subsection*{Применение к методу Ритца}

При использовании кусочно-линейных сплайнов в методе Ритца:

1. Приближённое решение: $u_N = \sum_{i=0}^{N} a_i \varphi_i(x)$

2. Система уравнений $\widehat{A} a = b$ имеет трёхдиагональную матрицу

3. Сходимость: $\|u - u_N\|_{W_2^1} \leq C h \|f\|_{L_2}$ (скорость $O(h)$)

4. По $L_2$-норме: $\|u - u_N\|_{L_2} \leq C h^2 \|f\|_{L_2}$ (скорость $O(h^2)$)

\subsection*{Сравнение с полиномами}

\begin{tabular}{|l|c|c|}
\hline
& \textbf{Кусочно-линейные} & \textbf{Глобальные полиномы} \\
\hline
Условие обусловленности & хорошо & плохо при $N \to \infty$ \\
Локальность поддержки & хорошо & нет \\
Простота & хорошо & сложновато \\
Требуемые знания & стандартные & специальные \\
\hline
\end{tabular}

\subsection*{Дополнительные пояснения}

\textbf{В.: Почему в $W_2^1$ норме сходимость $O(h)$, а в $L_2$ норме $O(h^2)$?}

О.: Потому что ошибка в производной больше, чем ошибка в самой функции. Это связано с регулярностью оператора и свойствами двойственности.

\textbf{В.: Как выбрать сетку для оптимальной аппроксимации?}

О.: Для гладких функций подходит равномерная сетка. Для функций с сингулярностями используют адаптивные сетки, сгущающиеся в проблемных областях.

\textbf{В.: Почему матрица системы трёхдиагональна?}

О.: Потому что носители соседних базисных функций перекрываются только с одним соседом слева и одним справа. Далёкие функции не коррелированы.

\textbf{В.: Как это обобщается на двумерные области?}

О.: На треугольной или четырёхугольной сетке используются билинейные (на прямоугольниках) или линейные (на треугольниках) базисные функции. Структура результирующей матрицы становится более сложной, но остаётся разреженной.
