\section{Билет 13. Третья краевая задача для уравнения Пуассона. Неравенство Фридрихса-Стеклова}

\subsection*{Постановка третьей краевой задачи}

Третья краевая задача для уравнения Пуассона:
\[
\boxed{\begin{cases}
-\Delta u = f(x) & \text{в } \Omega \\
\frac{\partial u}{\partial n} + \sigma u = g & \text{на } \partial\Omega
\end{cases}}
\]

где $\sigma \geq \sigma_0 > 0$ --- коэффициент граничного условия, $g$ --- граничные данные.

\subsection*{Вариационная формулировка}

Умножаем на пробную функцию $v$ и интегрируем:
\[
\int_\Omega \nabla u \cdot \nabla v \, d\Omega + \int_{\partial\Omega} \sigma u v \, dS = \int_\Omega f v \, d\Omega + \int_{\partial\Omega} g v \, dS
\]

Здесь граничное условие третьего рода естественно входит в вариационное уравнение.

\subsection*{Энергетическое пространство}

Так как граничное условие естественно:
\[
H_A = W_2^1(\Omega)
\]

без требования нулевых граничных значений.

Энергетическое скалярное произведение:
\[
[u, v]_A = \int_\Omega \nabla u \cdot \nabla v \, d\Omega + \int_{\partial\Omega} \sigma u v \, dS
\]

\subsection*{Обобщённая формулировка}

Обобщённым решением называется $u \in W_2^1(\Omega)$ такой, что:
\[
[u, v]_A = (f, v)_{L_2(\Omega)} + (g, v)_{L_2(\partial\Omega)} \quad \forall v \in W_2^1(\Omega)
\]

\subsection*{Положительная определённость}

Оператор положительно определён благодаря положительности граничного члена:
\[
[u, u]_A = \int_\Omega |\nabla u|^2 d\Omega + \int_{\partial\Omega} \sigma u^2 dS \geq \sigma_0 \int_{\partial\Omega} u^2 dS
\]

\subsection*{Неравенство Фридрихса-Стеклова}

\textbf{Теорема (неравенство Фридрихса-Стеклова):}

Для функции $u \in W_2^1(\Omega)$ выполняется:
\[
\boxed{\int_{\partial\Omega} u^2 dS \leq C_{FS} \left(\int_\Omega |\nabla u|^2 d\Omega + \int_\Omega u^2 d\Omega\right)}
\]

или в нормированной форме:
\[
\|u\|_{L_2(\partial\Omega)} \leq C_{FS} \|u\|_{W_2^1(\Omega)}
\]

\textit{Идея доказательства:}

Используя теорему о следе функций из $W_2^1(\Omega)$ и свойства граничных операторов, показывается, что граничные значения функции контролируются её нормой в $W_2^1$.

\subsection*{Следствие неравенства Фридрихса-Стеклова}

Из неравенства Фридрихса-Стеклова следует положительная определённость оператора третьей краевой задачи:
\[
[u, u]_A = \|\nabla u\|_{L_2}^2 + \sigma_0 \|u\|_{L_2(\partial\Omega)}^2 \geq \sigma_0 C_{FS}^{-2} \|u\|_{W_2^1}^2
\]

\subsection*{Сравнение неравенств}

Различие между основными неравенствами:

\begin{tabular}{|l|l|l|}
\hline
\textbf{Неравенство} & \textbf{Условия} & \textbf{Формулировка} \\
\hline
Фридрихса & $u|_{\partial\Omega} = 0$ & $\|u\|_{L_2} \leq C_F |\nabla u|_{L_2}$ \\
Пуанкаре & средн. значение = 0 & $\|u\|_{L_2} \leq C_P |\nabla u|_{L_2}$ \\
Фридрихса-Стеклова & общее случай & $|u|_{L_2(\partial\Omega)} \leq C_{FS} \|u\|_{W_2^1}$ \\
\hline
\end{tabular}

\subsection*{Метод Ритца для третьей краевой задачи}

1. Выбираем полную систему $\{\varphi_k\}$ в $W_2^1(\Omega)$ (без граничных условий)

2. Ищем приближение:
\[
u_n = \sum_{k=1}^{n} a_k \varphi_k
\]

3. Из условия:
\[
[u_n, \varphi_j]_A = (f, \varphi_j) + (g, \varphi_j)_{\partial\Omega}
\]

получаем систему уравнений для коэффициентов

4. Минимизируем энергетический функционал:
\[
J(u) = [u, u]_A - 2(f, u) - 2(g, u)_{\partial\Omega}
\]

\subsection*{Примеры физических приложений}

\textbf{Пример 1: Теплопроводность}

Уравнение Фурье с конвективным теплообменом на границе:
\[
-k \Delta T = Q, \quad -k \frac{\partial T}{\partial n} + h(T - T_\infty) = 0
\]

приводит к условию третьего рода $\frac{\partial T}{\partial n} + \frac{h}{k}T = \frac{h T_\infty}{k}$.

\textbf{Пример 2: Электростатика}

В области с диэлектриком и проводящей границей с сопротивлением:
\[
-\Delta \phi = \rho / \varepsilon, \quad \frac{\partial \phi}{\partial n} + \frac{1}{R}\phi = V_0
\]

\subsection*{Дополнительные пояснения}

\textbf{В.: Чем неравенство Фридрихса-Стеклова отличается от Фридрихса?}

О.: Фридрихса контролирует объёмную норму через градиент для функций с нулевыми граничными значениями. Фридрихса-Стеклова контролирует граничную норму через градиент и объёмную норму для функций без условий на границе.

\textbf{В.: Почему коэффициент $\sigma$ должен быть положительным?}

О.: Для обеспечения положительной определённости оператора. При $\sigma = 0$ получается задача Неймана, которая имеет ядро (константные функции).

\textbf{В.: Как в практике выбирают коэффициент $\sigma$?}

О.: В физических приложениях $\sigma$ зависит от свойств граничного материала (например, коэффициент теплоотдачи в теплопроводности). Его выбирают на основе физических данных.

\textbf{В.: Какова роль неравенства Фридрихса-Стеклова в численных методах?}

О.: Оно гарантирует, что энергетическая норма эквивалентна норме $W_2^1$, что необходимо для обоснования метода Ритца и анализа сходимости конечно-элементных схем.
