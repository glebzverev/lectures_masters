\section{Билет 21. Построение проекционной-сеточной схемы для краевой задачи для ОДУ 2 порядка. Анализ сходимости в $L_2$ и $W_2^1$}

\subsection*{Постановка задачи}

\[
\begin{cases}
-\frac{d}{dx}\left(p(x) \frac{du}{dx}\right) + q(x)u = f(x), & x \in (a, b) \\
u(a) = u(b) = 0
\end{cases}
\]

где $p(x) \geq p_0 > 0$, $q(x) \geq 0$, $f \in L_2(a, b)$.

\subsection*{Пространство энергии}

\[
H_A = \accentset{\circ}{W}_2^1(a, b) := \{u \in W_2^1(a, b) : u(a) = u(b) = 0\}
\]

\[
[u, v]_A = \int_a^b \left[p(x) u'(x) v'(x) + q(x) u(x) v(x)\right] dx
\]

\subsection*{Вариационная формулировка}

Найти $u \in H_A$ такой, что:
\[
[u, v]_A = (f, v)_{L_2} \quad \forall v \in H_A
\]

\subsection*{Проекционная-сеточная схема}

Вводим сетку на $[a, b]$ с шагом $h = (b - a)/N$:
\[
x_i = a + ih, \quad i = 0, 1, \ldots, N
\]

Конечномерное подпространство:
\[
H_h = \text{span}\{\varphi_1, \ldots, \varphi_{N-1}\}
\]

где $\varphi_i$ --- кусочно-линейные функции с $\varphi_i(x_j) = \delta_{ij}$.

Приближённое решение ищется в виде:
\[
u_h = \sum_{i=1}^{N-1} u_i \varphi_i(x)
\]

где $u_i = u_h(x_i)$ --- значения в узлах.

\subsection*{Дискретная система}

Вариационное уравнение на подпространстве:
\[
[u_h, \varphi_j]_A = (f, \varphi_j) \quad \forall j = 1, \ldots, N-1
\]

приводит к системе уравнений:
\[
\sum_{i=1}^{N-1} A_{ji} u_i = b_j, \quad j = 1, \ldots, N-1
\]

где:
\[
A_{ji} = [\varphi_i, \varphi_j]_A, \quad b_j = \int_a^b f(x) \varphi_j(x) dx
\]

\subsection*{Матричная структура для равномерной сетки}

При равномерной сетке и постоянных коэффициентах получается трёхдиагональная система:
\[
A_{jj} = \frac{2p}{h} + \frac{qh}{3}, \quad A_{j,j \pm 1} = -\frac{p}{h} + \frac{qh}{6}
\]

\subsection*{Теорема о сходимости в $W_2^1$}

\textbf{Теорема:}

Пусть $u \in W_2^2(a, b)$ --- точное решение, $u_h$ --- решение дискретной схемы. Тогда:
\[
\boxed{\|u - u_h\|_{W_2^1(a,b)} \leq C h \|u\|_{W_2^2(a,b)} \leq C h \|f\|_{L_2(a,b)}}
\]

\textit{Доказательство (идея):}

1. \textbf{Оптимальность в энергии:}
\[
|u - u_h|_A = \inf_{v_h \in H_h} |u - v_h|_A
\]

2. \textbf{Аппроксимация линейных функций:}
\[
\inf_{v_h} |u - v_h|_A \leq C h \|u\|_{W_2^2}
\]

(следует из аппроксимационных свойств кусочно-линейных функций).

3. \textbf{Эквивалентность норм:}
\[
|u|_A \sim \|u\|_{W_2^1}
\]

(следует из положительной определённости оператора и неравенства Фридрихса).

Объединяя эти факты, получаем требуемую оценку.

\subsection*{Теорема о сходимости в $L_2$}

\textbf{Теорема (двойственность):}

Под дополнительным условием регулярности (если $u \in W_2^2$ для любой $f \in L_2$):
\[
\boxed{\|u - u_h\|_{L_2(a,b)} \leq C h^2 \|f\|_{L_2(a,b)}}
\]

\textit{Доказательство (метод Nitsche):}

Рассмотрим вспомогательную задачу:
\[
-\frac{d}{dx}\left(p \frac{d\Phi}{dx}\right) + q\Phi = u - u_h
\]

По регулярности: $\|\Phi\|_{W_2^2} \leq C \|u - u_h\|_{L_2}$.

Затем:
\[
\|u - u_h\|_{L_2}^2 = [\Phi, u - u_h]_A = [\Phi - \Phi_h, u - u_h]_A \leq |\Phi - \Phi_h|_A \cdot |u - u_h|_A
\]

Оценивая обе нормы справа и используя $|u - u_h|_A \leq Ch \|f\|_{L_2}$, получаем результат.

\subsection*{Практическое значение оценок}

- Сходимость $O(h)$ в $W_2^1$ гарантирует надёжность вычисления производных
- Сходимость $O(h^2)$ в $L_2$ означает быстрое уменьшение глобальной ошибки
- Для достижения точности $\varepsilon$ требуется $h \sim \sqrt{\varepsilon}$, то есть $N \sim 1/\sqrt{\varepsilon}$

\subsection*{Адаптивные сетки}

Для области с неоднородным решением можно использовать адаптивные сетки, сгущающиеся там, где $|u''|$ велико. Это повышает скорость сходимости и снижает общий объём вычислений.

\subsection*{Дополнительные пояснения}

\textbf{В.: Почему в $L_2$ норме ошибка $O(h^2)$, а в $W_2^1$ норме только $O(h)$?}

О.: Потому что норма $W_2^1$ учитывает производные. Ошибка в производных больше, чем в самой функции. Это компенсируется благодаря регулярности оператора (метод двойственности).

\textbf{В.: Как выбрать размер сетки для требуемой точности?}

О.: Из оценки $\|u - u_h\|_{L_2} \leq C h^2 \|f\|$. Если требуется точность $10^{-4}$, и $\|f\| \approx 1$, то нужно $h^2 \leq 10^{-5}$, откуда $h \leq 0.003$, то есть примерно 300 узлов на отрезок $[0, 1]$.

\textbf{В.: Какова роль условия регулярности?}

О.: Регулярность (эллиптическая регулярность) гарантирует, что решение достаточно гладко ($u \in W_2^2$) при $f \in L_2$. Без этого $L_2$ оценка может быть хуже.

\textbf{В.: Можно ли использовать этот анализ для систем ОДУ?}

О.: Да, с соответствующей модификацией (матричные коэффициенты, векторные неизвестные). Основная идея остаётся прежней.
