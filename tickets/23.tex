\section{Билет 23. Задача на собственные значения для симметрического полуограниченного дифференциального оператора}

\subsection*{Постановка спектральной задачи}

Рассмотрим спектральную задачу в гильбертовом пространстве $H$:
\[
\boxed{A u = \lambda u, \quad u \in D(A), \quad u \neq 0}
\]

где $A$ --- симметричный оператор, полуограниченный снизу (то есть $\exists k \in \mathbb{R}$: $(Au, u) \geq k \|u\|^2$).

\subsection*{Пример: задача Штурма-Лиувилля}

\[
-\frac{d}{dx}\left(p(x) \frac{du}{dx}\right) + q(x) u = \lambda \rho(x) u, \quad u(a) = u(b) = 0
\]

где $\lambda$ --- собственное значение, $u$ --- собственная функция.

Эквивалентная вариационная постановка:
\[
\int_a^b \left[p(x) u'(x) v'(x) + q(x)u(x)v(x)\right] dx = \lambda \int_a^b \rho(x) u(x) v(x) dx
\]

\subsection*{Основные свойства собственных значений}

\textbf{Свойство 1: вещественность}

Для симметричного оператора все собственные значения вещественны.

\textit{Доказательство:} Если $Au = \lambda u$, то $(Au, u) = \lambda (u, u)$. Левая часть вещественна (симметричность), поэтому и $\lambda$ вещественна.

\textbf{Свойство 2: ортогональность собственных функций}

Собственные функции, соответствующие разным собственным значениям, ортогональны:
\[
\lambda_1 \neq \lambda_2 \Rightarrow (u_1, u_2) = 0
\]

\textit{Доказательство:} Из $Au_1 = \lambda_1 u_1$ и $Au_2 = \lambda_2 u_2$:
\[
\lambda_1 (u_1, u_2) = (Au_1, u_2) = (u_1, Au_2) = \lambda_2 (u_1, u_2)
\]

откуда $(\lambda_1 - \lambda_2)(u_1, u_2) = 0$, следовательно $(u_1, u_2) = 0$.

\textbf{Свойство 3: минимизация функционала Рэлея}

Первое (минимальное) собственное значение:
\[
\lambda_1 = \min_{u \neq 0} \frac{(Au, u)}{(u, u)} = \min_{u \neq 0} R(u)
\]

где $R(u)$ --- функционал Рэлея.

$n$-е собственное значение:
\[
\lambda_n = \min \left\{\frac{(Au, u)}{(u, u)} : u \perp u_1, \ldots, u_{n-1}\right\}
\]

\subsection*{Теорема о спектре положительно определённого оператора}

\textbf{Теорема:}

Пусть $A$ --- положительно определённый оператор в сепарабельном гильбертовом пространстве $H$. Тогда:

1. Спектр $A$ дискретен: $0 < \lambda_1 \leq \lambda_2 \leq \lambda_3 \leq \cdots \to \infty$

2. Существует полная ортонормированная система собственных функций:
\[
\{u_n\}_{n=1}^{\infty} : \quad Au_n = \lambda_n u_n, \quad (u_i, u_j) = \delta_{ij}
\]

3. Любой элемент $u \in H$ разлагается в ряд Фурье:
\[
u = \sum_{n=1}^{\infty} (u, u_n) u_n \quad \text{(сходится в норме } H \text{)}
\]

\subsection*{Мультипликативное неравенство}

Для собственных функций:
\[
(Au_n, u_n) = \lambda_n (u_n, u_n)
\]

Из положительной определённости:
\[
\lambda_n \geq \lambda_1 > 0
\]

Скорость роста собственных значений зависит от гладкости области и регулярности коэффициентов.

\subsection*{Приложения к краевым задачам}

\textbf{Пример 1: задача на отрезке}

Для $-u'' = \lambda u$, $u(0) = u(1) = 0$:
\[
\lambda_n = n^2 \pi^2, \quad u_n(x) = \sin(n\pi x)
\]

\textbf{Пример 2: задача на прямоугольнике}

Для $-\Delta u = \lambda u$ на $(0, a) \times (0, b)$ с условиями Дирихле:
\[
\lambda_{mn} = \pi^2 \left(\frac{m^2}{a^2} + \frac{n^2}{b^2}\right), \quad m, n = 1, 2, \ldots
\]

\subsection*{Асимптотика собственных значений}

По теореме Вейля для оператора Лапласа в размерности $m$:
\[
\lambda_n \sim \left(\frac{n}{V(m)}\right)^{2/m} \quad \text{при} \quad n \to \infty
\]

где $V(m)$ --- нормированный объём области.

\subsection*{Полнота системы собственных функций}

Ключевой вопрос: образует ли система $\{u_n\}$ полную систему в пространстве $H$?

\textbf{Ответ:} Для положительно определённого оператора в сепарабельном пространстве --- да.

Это позволяет развивать теорию рядов Фурье по собственным функциям.

\subsection*{Дополнительные пояснения}

\textbf{В.: Что означает "полуограниченный снизу"?}

О.: Означает, что оператор ограничен снизу константой $k$, но может быть не ограничен сверху. Для положительно определённого оператора $k > 0$.

\textbf{В.: Почему полнота системы важна?}

О.: Потому что она гарантирует, что любое решение можно представить в виде ряда по собственным функциям. Это основа спектральных методов.

\textbf{В.: Как используются собственные функции в численных методах?}

О.: В методе конечных элементов с собственными функциями в качестве базиса можно достичь очень высокой точности. Они образуют естественный базис для оператора.

\textbf{В.: Почему скорость роста $\lambda_n$ зависит от размерности?}

О.: Потому что при увеличении размерности область объёма $V$ "становится тоньше" в некотором смысле, что приводит к более быстрому росту собственных значений.
