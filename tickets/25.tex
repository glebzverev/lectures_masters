\section{Билет 25. Метод Ритца в проблеме вычисления собственных значений задачи Дирихле}

\subsection*{Постановка задачи на собственные значения}

Рассмотрим задачу Дирихле для оператора Лапласа:
\[
-\Delta u = \lambda u, \quad u|_{\partial\Omega} = 0
\]

или в вариационной форме:
\[
\lambda = \frac{\int_\Omega |\nabla u|^2 d\Omega}{\int_\Omega u^2 d\Omega}
\]

\subsection*{Метод Ритца для спектральной задачи}

Вместо точного пространства $H_A = \accentset{\circ}{W}_2^1(\Omega)$, выбираем конечномерное подпространство:
\[
H_n = \text{span}\{\varphi_1, \varphi_2, \ldots, \varphi_n\}
\]

Ищем приближённое решение в виде:
\[
u_n = \sum_{k=1}^{n} a_k \varphi_k
\]

\subsection*{Вариационная формулировка на подпространстве}

Функционал Рэлея на подпространстве:
\[
R_n(u_n) = \frac{\sum_{i,j} a_i a_j [\varphi_i, \varphi_j]_A}{\sum_{i,j} a_i a_j (\varphi_i, \varphi_j)}
\]

Минимизация даёт обобщённую спектральную задачу:
\[
\boxed{\widehat{A} a = \lambda \widehat{M} a}
\]

где:
\[
\widehat{A}_{ij} = [\varphi_i, \varphi_j]_A = \int_\Omega \nabla \varphi_i \cdot \nabla \varphi_j d\Omega
\]

\[
\widehat{M}_{ij} = (\varphi_i, \varphi_j)_{L_2} = \int_\Omega \varphi_i \varphi_j d\Omega
\]

\subsection*{Матричный вид дискретной задачи}

Дискретная задача на собственные значения:
\[
\widehat{A} a = \lambda \widehat{M} a
\]

Здесь:
- $\widehat{A}$ --- матрица жёсткости (stiffness matrix)
- $\widehat{M}$ --- матрица масс (mass matrix)
- $\lambda$ --- приближённое собственное значение
- $a$ --- компоненты приближённой собственной функции

\subsection*{Свойства матриц}

1. \textbf{Симметрия:} обе матрицы симметричны

2. \textbf{Положительная определённость:} $\widehat{A}$ и $\widehat{M}$ положительно определены

3. \textbf{Разреженность:} особенно для конечно-элементных базисов

4. \textbf{Обусловленность:} число обусловленности $\widehat{A}$ и $\widehat{M}$ растёт с уменьшением $h$

\subsection*{Решение дискретной задачи}

Обобщённую спектральную задачу можно свести к стандартной:
\[
\widehat{M}^{-1} \widehat{A} a = \lambda a
\]

или через разложение Холецкого $\widehat{M} = L L^T$:
\[
L^{-T} \widehat{A} L^{-1} b = \lambda b, \quad a = L^{-1} b
\]

Для вычисления собственных значений используют:
- Метод степенных итераций (для первого собственного значения)
- QR-алгоритм (для нескольких собственных значений)
- Методы Ланцоша или Арнольди (для больших разреженных матриц)

\subsection*{Теорема о сходимости приближённых собственных значений}

\textbf{Теорема:}

Пусть $\lambda_k^{(h)}$ --- $k$-е собственное значение дискретной задачи, полученное методом Ритца с сетку с параметром $h$. Тогда при $h \to 0$:

\[
\boxed{\lambda_k^{(h)} \to \lambda_k \quad \text{и} \quad |\lambda_k^{(h)} - \lambda_k| \leq C_k h^2}
\]

где $\lambda_k$ --- точное собственное значение, $C_k$ --- константа, зависящая от $k$.

\textit{Идея доказательства:}

1. Фиксируем $k$ и рассматриваем подпространство, натянутое на первые $k$ собственных функций точной задачи

2. Функция Рэлея на этом подпространстве минимизируется на $\lambda_k$

3. При уменьшении $h$ пространство $H_n$ приближается к точному пространству, и минимум функции Рэлея приближается к $\lambda_k$

4. Скорость сходимости зависит от аппроксимационных свойств подпространства (обычно $O(h^2)$ в $L_2$ для линейных элементов)

\subsection*{Сходимость собственных функций}

Наряду с собственными значениями, собственные функции также сходятся:
\[
\|u_k - u_k^{(h)}\|_{L_2(\Omega)} \leq C h^2 \|u_k\|_{W_2^2(\Omega)}
\]

где $u_k^{(h)}$ --- приближённая собственная функция, $u_k$ --- точная.

\subsection*{Практическая реализация}

\textbf{Алгоритм метода Ритца:}

1. Построить сетку на области $\Omega$ (триангуляция)

2. Выбрать пространство конечных элементов $H_n$ (например, кусочно-линейные функции)

3. Вычислить матрицы жёсткости и масс:
\[
A_{ij} = \int_\Omega \nabla \varphi_i \cdot \nabla \varphi_j d\Omega, \quad M_{ij} = \int_\Omega \varphi_i \varphi_j d\Omega
\]

4. Решить обобщённую спектральную задачу $\widehat{A} a = \lambda \widehat{M} a$

5. Интерпретировать решение: собственные значения $\lambda$ и собственные функции $u_n = \sum_k a_k \varphi_k$

\subsection*{Примеры приложений}

\begin{enumerate}
\item \textbf{Нахождение частот собственных колебаний} мембраны или пластины

\item \textbf{Анализ устойчивости} в проблемах конвекции Рэлея-Бенара

\item \textbf{Вычисление критических нагрузок} при потере устойчивости конструкции

\item \textbf{Определение граничных условий} энергетических зон в квантовой механике (зонная теория)
\end{enumerate}

\subsection*{Численные особенности}

1. \textbf{Плохая обусловленность:} при уменьшении $h$ число обусловленности матриц растёт как $O(1/h^2)$

2. \textbf{Кратные собственные значения:} численные методы могут дать близкие, но различные приближения для кратных значений

3. \textbf{Разреженность:} использование структуры разреженности критично для больших задач

4. \textbf{Адаптивность:} адаптивные сетки могут улучшить точность вычисления отдельных собственных значений

\subsection*{Дополнительные пояснения}

\textbf{В.: Почему скорость сходимости $O(h^2)$, а не $O(h)$?}

О.: Это следует из того, что функционал Рэлея имеет критические точки в собственных значениях. Погрешность в функционале убывает как $O(h^2)$, что приводит к такой же скорости для собственных значений.

\textbf{В.: Как выбрать размер конечно-элементной сетки?}

О.: Из требуемой точности. Если нужна точность $10^{-6}$ для $n$-го собственного значения, то $h \sim 10^{-3}$ (примерно). На практике используют апостериорные оценки.

\textbf{В.: Может ли метод Ритца пропустить собственное значение?}

О.: Нет, при условии полноты системы базисных функций. Все собственные значения, не превышающие некоторого уровня, будут найдены, хотя и с погрешностью.

\textbf{В.: Какой метод решения спектральной задачи лучше всего?}

О.: Зависит от цели: метод степенных итераций для первого собственного значения, QR для нескольких, Ланцоша для много собственных значений большой разреженной матрицы.
