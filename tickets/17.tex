\section{Билет 17. Степенные сплайны нулевой степени дефекта 1. Теоремы аппроксимации}

\subsection*{Определение степенного сплайна}

На отрезке $[a, b]$ введём сетку узлов:
\[
a = x_0 < x_1 < \cdots < x_N = b
\]

\textbf{Определение:} Сплайн степени $n$ дефекта $\nu$ (обозначение $S_{n,\nu}$) --- функция, которая на каждом подотрезке $[x_i, x_{i+1}]$ является полиномом степени $n$, а во всех внутренних узлах имеет непрерывные производные до порядка $n - \nu$ включительно.

\subsection*{Сплайны нулевой степени дефекта 1}

Сплайн нулевой степени дефекта 1 обозначается $S_{0,1}$ --- это кусочно постоянная функция:
\[
S_{0,1}(x) = \begin{cases}
c_i & x \in (x_{i-1}, x_i), \quad i = 1, \ldots, N
\end{cases}
\]

Дефект 1 означает непрерывность нулевого порядка, то есть саму функцию можно разрывать в узлах.

\subsection*{Полнота и линейная независимость}

Пространство сплайнов $S_{0,1}$:
\[
\mathcal{S}_{0,1} = \{s(x) : s \text{ --- сплайн степени 0 дефекта 1}\}
\]

имеет размерность $N$ (число подотрезков). Базис:
\[
\varphi_i(x) = \begin{cases}
1 & x \in (x_{i-1}, x_i) \\
0 & x \notin (x_{i-1}, x_i)
\end{cases}, \quad i = 1, \ldots, N
\]

Функции $\{\varphi_1, \ldots, \varphi_N\}$ линейно независимы.

\subsection*{Ортогональность в $L_2$}

Базисные функции ортогональны в смысле $L_2$:
\[
(\varphi_i, \varphi_j) = \begin{cases}
h_i & i = j \\
0 & i \neq j
\end{cases}
\]

где $h_i = x_i - x_{i-1}$ --- длина $i$-го подотрезка.

\subsection*{Теоремы аппроксимации}

\textbf{Теорема 1 (аппроксимация в $L_p$):}

Для функции $u \in W_p^1(a, b)$ существует элемент $s \in \mathcal{S}_{0,1}$ такой, что:
\[
\|u - s\|_{L_p(a,b)} \leq C h \|u'\|_{L_p(a,b)}
\]

где $C$ --- константа, не зависящая от $h$ и $u$, и $h = \max_i h_i$.

\textit{Доказательство (идея):} На каждом подотрезке выбираем $s(x) = \bar{u}_i$ (среднее значение $u$ на подотрезке). Затем оцениваем отклонение:
\[
|u(x) - \bar{u}_i| \leq \frac{1}{h_i} \int_{x_{i-1}}^{x_i} |u(x) - u(\xi)| d\xi \leq h_i \|u'\|_{L_\infty}
\]

Интегрируя и применяя неравенство Гельдера, получаем оценку.

\textbf{Теорема 2 (аппроксимация в $C(\Omega)$):}

Для функции $u \in C^1[a, b]$ существует $s \in \mathcal{S}_{0,1}$ такой, что:
\[
\|u - s\|_{C(a,b)} \leq C h \|u'\|_{C(a,b)}
\]

\subsection*{Оптимальность оценок}

Полученные оценки $O(h)$ оптимальны для сплайнов нулевой степени. Для лучшей аппроксимации нужны сплайны более высокой степени (кусочно-линейные, квадратичные и т.д.).

\subsection*{Пространство сплайнов в энергетическом методе}

Для применения метода Ритца с кусочно-постоянными сплайнами нужно использовать их в роли пробных функций для вариационных уравнений. Однако, поскольку сплайны нулевой степени не дифференцируемы, их прямое использование затруднено.

Более практичны следующие сплайны дефекта 0 (непрерывные) для вариационных методов.

\subsection*{Расширенное пространство базиса}

Используя оператор масс-консистентного проектирования, можно работать с кусочно-постоянными функциями в численных схемах (например, в методе конечных объёмов).

\subsection*{Дополнительные пояснения}

\textbf{В.: Почему дефект = 1 для кусочно-постоянной функции?}

О.: Потому что непрерывность нулевого порядка означает непрерывность самой функции. Дефект 1 означает, что имеем непрерывность на один порядок ниже, то есть функция может быть разрывной.

\textbf{В.: Где используются кусочно-постоянные сплайны?}

О.: В методе конечных объёмов, в аппроксимации разрывных функций, в методах для уравнений гиперболического типа.

\textbf{В.: Почему оценка аппроксимации $O(h)$, а не лучше?}

О.: Потому что сплайн степени 0 --- это локально постоянная функция. Чтобы приблизить гладкую функцию с точностью $O(h^k)$, нужен сплайн степени не менее $k-1$.

\textbf{В.: Как улучшить точность аппроксимации?}

О.: Использовать сплайны более высокой степени: линейные (степень 1), квадратичные (степень 2) и т.д. Кусочно-линейные сплайны дают точность $O(h^2)$ в $L_2$ и $O(h)$ в норме $W_2^1$.
