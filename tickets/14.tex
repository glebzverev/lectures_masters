\section{Билет 14. Задача Неймана для уравнения Пуассона. Неравенство Пуанкаре}

\subsection*{Постановка задачи Неймана}

Задача Неймана для уравнения Пуассона:
\[
\boxed{\begin{cases}
-\Delta u = f(x) & \text{в } \Omega \\
\frac{\partial u}{\partial n} = g & \text{на } \partial\Omega
\end{cases}}
\]

где $g$ --- заданная нормальная производная на границе.

\subsection*{Условие совместности}

В отличие от задачи Дирихле, задача Неймана имеет решение не для всякой правой части.

\textbf{Теорема (условие совместности):}

Для существования решения необходимо выполнение условия:
\[
\int_\Omega f(x) d\Omega + \int_{\partial\Omega} g(x) dS = 0
\]

\textit{Доказательство:} Применяя формулу Грина к уравнению и условию Неймана:
\[
\int_\Omega (-\Delta u) \, d\Omega = \int_{\partial\Omega} \frac{\partial u}{\partial n} dS
\]

откуда $\int_\Omega f \, d\Omega = \int_{\partial\Omega} g \, dS$. $\square$

\subsection*{Решение с точностью до константы}

Множество решений задачи Неймана образует аффинное подпространство размерности 1: если $u$ --- одно решение, то всякое решение имеет вид $u + c$ для некоторой константы $c$.

Это отражает физическую интерпретацию: для многих задач (диффузия, электростатика) потенциал определён с точностью до произвольной постоянной.

\subsection*{Нормализованная задача Неймана}

Обычно задачу Неймана нормализуют, добавляя условие:
\[
\int_\Omega u \, d\Omega = 0 \quad \text{или} \quad u(x_0) = 0 \text{ в точке } x_0
\]

После нормализации решение становится единственным.

\subsection*{Вариационная формулировка}

Умножаем на пробную функцию и интегрируем:
\[
\int_\Omega \nabla u \cdot \nabla v \, d\Omega = \int_\Omega f v \, d\Omega + \int_{\partial\Omega} g v \, dS
\]

Здесь граничное условие естественно входит в правую часть.

\subsection*{Энергетическое пространство}

\[
H_A = W_2^1(\Omega) / \mathbb{R}
\]

то есть пространство функций из $W_2^1(\Omega)$ с условием нормировки $\int_\Omega u \, d\Omega = 0$.

Энергетическое скалярное произведение:
\[
[u, v]_A = \int_\Omega \nabla u \cdot \nabla v \, d\Omega
\]

\subsection*{Неравенство Пуанкаре}

\textbf{Теорема (неравенство Пуанкаре):}

Для функции $u \in W_2^1(\Omega)$ с условием нулевого среднего значения:
\[
\int_\Omega u \, d\Omega = 0
\]

выполняется:
\[
\boxed{\|u\|_{L_2(\Omega)} \leq C_P \|\nabla u\|_{L_2(\Omega)}}
\]

где константа Пуанкаре $C_P$ зависит от диаметра области $\Omega$.

Для шара радиуса $R$: $C_P = \frac{R}{\sqrt{\lambda_1}}$, где $\lambda_1$ --- первое собственное значение оператора Лапласа.

\subsection*{Доказательство неравенства Пуанкаре}

\textit{Идея:} Если $u$ имеет нулевое среднее, то функция не может быть константой, и её значения существенно варьируются по области. Интуитивно, градиент должен быть больше.

Формально, разложим $u$ в ряд по собственным функциям оператора Лапласа:
\[
u = \sum_{k=2}^{\infty} c_k \varphi_k
\]

(первое слагаемое соответствует константе). Тогда:
\[
\|u\|_{L_2}^2 = \sum_{k=2}^{\infty} c_k^2
\]

\[
\|\nabla u\|_{L_2}^2 = \sum_{k=2}^{\infty} \lambda_k c_k^2 \geq \lambda_2 \|u\|_{L_2}^2
\]

откуда $\|u\|_{L_2} \leq \frac{1}{\sqrt{\lambda_2}} \|\nabla u\|_{L_2}$.

\subsection*{Положительная определённость оператора Неймана}

При нормировке $\int_\Omega u \, d\Omega = 0$, оператор становится положительно определённым:
\[
[u, u]_A = \|\nabla u\|_{L_2}^2 \geq \lambda_2 \|u\|_{L_2}^2
\]

где $\lambda_2$ --- второе собственное значение оператора Лапласа.

\subsection*{Метод Ритца для задачи Неймана}

1. Выбираем базисные функции $\{\varphi_k\}$ с условием нулевого среднего:
\[
\int_\Omega \varphi_k d\Omega = 0
\]

2. Ищем приближение $u_n = \sum_{k=1}^{n} a_k \varphi_k$

3. Система уравнений:
\[
\sum_{k=1}^{n} A_{jk} a_k = b_j
\]

где $A_{jk} = [\varphi_k, \varphi_j]_A$, $b_j = (f, \varphi_j) + (g, \varphi_j)_{\partial\Omega}$

4. Минимизируем функционал:
\[
J(u) = \|\nabla u\|_{L_2}^2 - 2(f, u) - 2(g, u)_{\partial\Omega} \to \min
\]

на подпространстве с нулевым средним.

\subsection*{Примеры приложений}

\textbf{Пример 1: Течение несжимаемой жидкости}

Уравнение для давления:
\[
-\Delta p = \nabla \cdot f
\]

с условием Неймана на твёрдой границе (нормальная компонента скорости равна нулю, что даёт $\partial p / \partial n = 0$ на части границы).

\textbf{Пример 2: Диффузия с замкнутой системой}

Концентрация вещества в замкнутой системе:
\[
-D \Delta c = r
\]

с условием Неймана на границе (поток = 0): $\partial c / \partial n = 0$.

\subsection*{Дополнительные пояснения}

\textbf{В.: Почему нужно условие совместности для задачи Неймана?}

О.: Потому что функция с нулевым нормальным потоком на замкнутой границе может измениться только в результате источников внутри. Если источников нет ($f = 0$) и потока нет ($g = 0$), то функция константа.

\textbf{В.: Как практически обеспечить нулевое среднее?}

О.: Либо выбирать базисные функции с нулевым средним, либо добавить множитель Лагранжа (способ штрафа) в функционал.

\textbf{В.: В чём отличие констант в неравенствах Фридрихса и Пуанкаре?}

О.: Фридрихса: константа связана с линейным размером области. Пуанкаре: константа связана с первым ненулевым собственным значением. Последнее часто меньше, что означает более сильное неравенство.

\textbf{В.: Почему неравенство Пуанкаре сильнее для функций с нулевым средним?}

О.: Потому что условие нулевого среднего исключает константную функцию, которая является критической для оценок норм.
