\section{Билет 9. Основные свойства обобщённого решения задачи Дирихле}

\subsection*{Принадлежность пространству $H^1$}

\textbf{Теорема:}

Обобщённое решение $u$ краевой задачи Дирихле для уравнения Штурма-Лиувилля принадлежит пространству Соболева $W_2^1(a, b)$:
\[
u \in W_2^1(a, b) = H^1(a, b)
\]

\textit{Доказательство:}

Из вариационного уравнения:
\[
[u, v]_A = (f, v)_{L_2} \quad \forall v \in H_A
\]

следует:
\[
\int_a^b p(x) u'(x) v'(x) dx + \int_a^b q(x) u(x) v(x) dx = \int_a^b f(x) v(x) dx
\]

Из этого уравнения видно, что:
\[
\int_a^b p(x) (u')^2 dx + \int_a^b q(x) u^2 dx = [u, u]_A < \infty
\]

следовательно $u' \in L_2(a,b)$. Поскольку $u$ также интегрируемо с квадратом (из энергетической нормы), получаем $u \in W_2^1(a,b)$.

\subsection*{Обобщённая производная}

\textbf{Определение:} Функция $v \in L_2(\Omega)$ называется \textbf{обобщённой производной} функции $u \in L_2(\Omega)$ (в смысле распределений), если:
\[
\int_\Omega u \frac{\partial \psi}{\partial x_i} dx = -\int_\Omega v \psi dx \quad \forall \psi \in C_0^\infty(\Omega)
\]

где $C_0^\infty(\Omega)$ --- гладкие функции с компактным носителем.

\textbf{Свойства:}

1. Обобщённая производная (если существует) единственна (почти всюду)

2. Обобщённая производная совпадает с классической, если классическая существует

3. Для решения $u \in H^1(a,b)$ обобщённая производная $u'$ принадлежит $L_2(a,b)$

\subsection*{Граничные следы}

Важное свойство функций из $W_2^1(\Omega)$ --- существование \textbf{следа} на границе.

\textbf{Теорема о следе:}

Для функции $u \in W_2^1(\Omega)$ существует функция $\gamma_0(u) \in L_2(\partial\Omega)$ (граничный след) такая, что:
\[
\gamma_0(u) = u|_{\partial\Omega} \quad \text{для } u \in C(\overline{\Omega})
\]

\subsection*{Пространство $\accentset{\circ}{W}_2^1(\Omega)$}

Пространство функций с нулевыми граничными условиями определяется как:
\[
\accentset{\circ}{W}_2^1(\Omega) = \{u \in W_2^1(\Omega) : u|_{\partial\Omega} = 0\}
\]

Это замкнутое подпространство $W_2^1(\Omega)$.

\subsection*{Нормы в пространстве $W_2^1$}

Существуют три эквивалентные нормы:

1. \textbf{Обычная норма:}
\[
\|u\|_{W_2^1} = \sqrt{\|u\|_{L_2}^2 + \|u'\|_{L_2}^2}
\]

2. \textbf{Энергетическая норма:}
\[
|u|_A = \sqrt{\int_a^b [p(x)(u')^2 + q(x)u^2] dx}
\]

3. \textbf{Неравенство Фридрихса:}
Для $u \in \accentset{\circ}{W}_2^1(\Omega)$:
\[
\|u\|_{L_2} \leq C \|u'\|_{L_2}
\]

так что норма может быть определена как $\|u\| = \|u'\|_{L_2}$.

\subsection*{Регулярность решения}

\textbf{Теорема о регулярности:}

Если правая часть $f \in L_2(a, b)$, то обобщённое решение $u$ задачи Штурма-Лиувилля принадлежит $W_2^2(a, b)$ и удовлетворяет дифференциальному уравнению почти всюду:
\[
-\frac{d}{dx}\left(p(x) \frac{du}{dx}\right) + q(x)u = f(x) \quad \text{п.в.}
\]

\subsection*{Примеры}

\textbf{Пример 1: Уравнение теплопроводности в стационарном виде}

\[
-u''(x) = f(x), \quad u(0) = u(1) = 0
\]

Обобщённое решение лежит в $H^1(0,1)$ и удовлетворяет интегральному тождеству:
\[
\int_0^1 u'(x) v'(x) dx = \int_0^1 f(x) v(x) dx \quad \forall v \in \accentset{\circ}{H}^1(0,1)
\]

\textbf{Пример 2: Уравнение с переменными коэффициентами}

\[
-\frac{d}{dx}\left(e^x \frac{du}{dx}\right) = \sin(\pi x), \quad u(0) = u(1) = 0
\]

Энергетическое скалярное произведение:
\[
[u,v]_A = \int_0^1 e^x u'(x) v'(x) dx
\]

\subsection*{Дополнительные пояснения}

\textbf{В.: Как формально определяется обобщённая производная?}

О.: Через интегральное тождество (выше). Интуитивно: это предельное поведение разностных частных при стремлении приращения к нулю.

\textbf{В.: Почему важна обобщённая производная?}

О.: Потому что при работе с обобщёнными решениями функции могут быть недифференцируемы в классическом смысле, но иметь обобщённую производную. Это позволяет распространить методы анализа на более широкий класс функций.

\textbf{В.: Как проверить, что функция из $W_2^1$?}

О.: Нужно показать, что функция и её производная (в смысле распределений) интегрируемы с квадратом. Практически это часто следует из интегральных неравенств (Фридрихса, Соболева и т.д.).

\textbf{В.: Верно ли, что любая функция из $C^1$ принадлежит $W_2^1$?}

О.: Да, но обратное неверно. $W_2^1$ содержит функции, которые не непрерывны в классическом смысле.
