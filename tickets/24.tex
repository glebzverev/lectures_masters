\section{Билет 24. Вариационная постановка задачи на собственные значения симметричного положительного операторного уравнения. Приложение к основным задачам математической физики}

\subsection*{Обобщённая спектральная задача}

Рассмотрим обобщённую вариационную задачу на собственные значения:
\[
\boxed{A u = \lambda B u}
\]

где:
- $A$ --- симметричный, положительно определённый оператор
- $B$ --- симметричный, положительно определённый оператор
- $D(A) \subset D(B)$

\subsection*{Вариационная формулировка}

В обозначениях энергетических норм задача эквивалентна:
\[
\frac{(Au, u)}{(Bu, u)} \to \text{экстремум}
\]

Первое собственное значение:
\[
\lambda_1 = \min_{u \neq 0} \frac{(Au, u)}{(Bu, u)}
\]

$n$-е собственное значение (определяется минимизацией на ортогональном дополнении к первым $n-1$ собственным функциям в смысле скалярного произведения $(B \cdot, \cdot)$):
\[
\lambda_n = \min \left\{\frac{(Au, u)}{(Bu, u)} : (Bu, u_k) = 0, k = 1, \ldots, n-1\right\}
\]

\subsection*{Пример 1: Задача Штурма-Лиувилля}

\[
-\frac{d}{dx}\left(p(x) \frac{du}{dx}\right) + q(x) u = \lambda \rho(x) u
\]

Здесь:
\[
(Au, v) = \int_a^b \left[p u' v' + q u v\right] dx, \quad (Bu, v) = \int_a^b \rho u v dx
\]

Функция $\rho(x)$ --- весовая функция (плотность материала, теплоёмкость и т.д.).

Вариационная постановка:
\[
\lambda = \frac{\int_a^b [p(u')^2 + qu^2] dx}{\int_a^b \rho u^2 dx}
\]

\subsection*{Пример 2: Колебания мембраны}

На области $\Omega \subset \mathbb{R}^2$:
\[
-\Delta u = \lambda \rho(x) u, \quad u|_{\partial\Omega} = 0
\]

где $\rho(x)$ --- плотность мембраны.

Вариационная форма:
\[
\lambda = \frac{\int_\Omega |\nabla u|^2 d\Omega}{\int_\Omega \rho u^2 d\Omega}
\]

Собственные значения --- квадраты частот колебаний, собственные функции --- формы колебаний.

\subsection*{Пример 3: Задача на собственные значения Штурма-Лиувилля 4-го порядка}

Уравнение колебаний балки:
\[
\frac{d^2}{dx^2}\left(EI \frac{d^2u}{dx^2}\right) = \lambda \rho u
\]

с граничными условиями (защемление на концах).

Вариационная постановка:
\[
\lambda = \frac{\int_0^L EI (u'')^2 dx}{\int_0^L \rho u^2 dx}
\]

\subsection*{Пример 4: Волноводный уравнение}

Для распространения волн в волноводе:
\[
-\Delta_\perp u - \frac{d^2u}{dz^2} = \lambda u
\]

Разделение переменных $u(x, y, z) = U(x, y) e^{i\lambda z}$ даёт:
\[
-\Delta_\perp U = (\lambda - \omega^2) U
\]

где $\lambda$ --- дисперсионное соотношение.

\subsection*{Свойства спектра обобщённой задачи}

\textbf{Теорема:}

Для обобщённой спектральной задачи с положительно определёнными $A$ и $B$:

1. Все собственные значения положительны: $\lambda_n > 0$

2. Собственные функции образуют $B$-ортонормированный базис:
\[
(Bu_i, u_j) = \delta_{ij}
\]

3. Спектр дискретен и стремится к бесконечности:
\[
0 < \lambda_1 \leq \lambda_2 \leq \cdots \to \infty
\]

4. Любой элемент разлагается в ряд по собственным функциям (в смысле $A$-нормы)

\subsection*{Применение к физическим задачам}

\begin{tabular}{|l|l|l|}
\hline
\textbf{Задача} & \textbf{Оператор} A & \textbf{Оператор} B \\
\hline
Продольные колебания & $p u' v'$ & $\rho u v$ \\
Поперечные колебания & $EI u'' v''$ & $\rho u v$ \\
Волны в струне & $T u' v'$ & $\rho u v$ \\
Волны в среде & $\int \nabla u \cdot \nabla v$ & $\int c^2 u v$ \\
\hline
\end{tabular}

где $p, \rho$ --- коэффициенты, $EI$ --- жёсткость, $T$ --- натяжение, $c$ --- скорость волны.

\subsection*{Дополнительные пояснения}

\textbf{В.: Почему функция Рэлея имеет именно такой вид?}

О.: Потому что его критические точки --- это в точности обобщённые собственные значения и собственные функции. Это глубокая связь между оптимизацией и спектральной теорией.

\textbf{В.: Как практически вычислить несколько первых собственных значений?}

О.: Методом Ритца: аппроксимировать подпространством конечной размерности и решить получившуюся конечномерную обобщённую задачу на собственные значения.

\textbf{В.: Почему весовая функция $\rho$ важна?}

О.: Потому что она отражает физические свойства системы (плотность, теплоёмкость и т.д.). Различные $\rho$ приводят к разным спектрам.

\textbf{В.: Как выбрать начальное приближение для итеративного поиска собственного значения?}

О.: Использовать формулу Рэлея с хорошим начальным приближением к собственной функции. Метод степенных итераций сходится к доминирующему собственному значению.
