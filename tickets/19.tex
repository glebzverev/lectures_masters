\section{Билет 19. Применение сплайн-аппроксимации в методе Ритца для получения приближённых решений}

\subsection*{Интеграция сплайнов и метода Ритца}

Метод Ритца с кусочно-линейными сплайнами представляет собой практический численный метод для решения краевых задач. Это фактически приводит к методу конечных элементов в его простейшей форме.

\subsection*{Алгоритм}

\textbf{Шаг 1: Дискретизация области}

На отрезке $[a, b]$ введём сетку:
\[
a = x_0 < x_1 < \cdots < x_N = b, \quad h_i = x_i - x_{i-1}
\]

\textbf{Шаг 2: Выбор базиса}

Используем кусочно-линейные базисные функции:
\[
\varphi_i(x), \quad i = 0, 1, \ldots, N
\]

с условием граничных условий (для задачи Дирихле $u(a) = u(b) = 0$ берём $\varphi_0, \varphi_N$ не в базис, или используем подпространство с нулевыми значениями на границе).

\textbf{Шаг 3: Построение системы}

Рассмотрим вариационное уравнение:
\[
[u_N, \varphi_j]_A = (f, \varphi_j) \quad \forall j
\]

где $u_N = \sum_{k} a_k \varphi_k$.

Получается система:
\[
\sum_k A_{jk} a_k = b_j
\]

\textbf{Шаг 4: Сборка матрицы}

Элементы матрицы вычисляются как:
\[
A_{jk} = [\varphi_k, \varphi_j]_A = \int_a^b \left[p(x) \varphi_k' \varphi_j' + q(x) \varphi_k \varphi_j\right] dx
\]

Интегралы можно вычислять точно (для простых коэффициентов) или численно (квадратурные формулы).

\textbf{Шаг 5: Вычисление правой части}

\[
b_j = (f, \varphi_j) = \int_a^b f(x) \varphi_j(x) dx
\]

\textbf{Шаг 6: Решение системы}

Матрица $A$ трёхдиагональна, можно решить быстрым алгоритмом прогонки (O(N) операций).

\textbf{Шаг 7: Интерпретация решения}

Коэффициенты $a_k$ --- это приближённые значения $u_N(x_k)$ в узлах сетки.

\subsection*{Практический пример: задача Штурма-Лиувилля}

Рассмотрим задачу:
\[
-u'' = f, \quad u(0) = u(1) = 0
\]

\textbf{Энергетическое скалярное произведение:}
\[
[u, v]_A = \int_0^1 u'v' dx
\]

\textbf{Матричные элементы для равномерной сетки с шагом $h = 1/N$:}

На подотрезке $[x_i, x_{i+1}]$:
\[
\int_{x_i}^{x_{i+1}} \varphi_i' \varphi_i' dx = \int_{x_i}^{x_{i+1}} \frac{1}{h^2} dx = \frac{1}{h}
\]

Вычисляя все интегралы, получаем классическую трёхдиагональную матрицу:
\[
A = \frac{1}{h} \begin{pmatrix}
2 & -1 & 0 & \cdots \\
-1 & 2 & -1 & \cdots \\
0 & -1 & 2 & \cdots \\
\vdots & \vdots & \vdots & \ddots
\end{pmatrix}
\]

\textbf{Правая часть:}

Для интегралов $(f, \varphi_i)$ часто используют квадратурные формулы.

\subsection*{Оценки погрешности}

Для решения $u \in W_2^2$ и приближения $u_N$:
\[
\|u - u_N\|_{W_2^1} \leq C h \|u\|_{W_2^2}
\]

Умножая на константу регулярности оператора и норму источника, можно получить:
\[
\|u - u_N\|_{W_2^1} \leq C h \|f\|_{L_2}
\]

\subsection*{Вычислительные особенности}

1. \textbf{Локальность вычислений:} матрица редко заполнена, можно организовать эффективное хранение

2. \textbf{Простота сборки:} матрица собирается из локальных вкладов каждого элемента

3. \textbf{Условие совместимости размеров:} норма матрицы растёт как $O(1/h^2)$, число обусловленности как $O(1/h^2)$

4. \textbf{Сходимость:} при уменьшении $h$ решение сходится к точному

\subsection*{Расширение на системы ОДУ}

Для систем уравнений базисные функции принимают векторные значения, но процедура остаётся принципиально той же.

\subsection*{Переход к двумерным задачам}

На прямоугольной или треугольной сетке используются билинейные (на прямоугольниках) или линейные (на треугольниках) базисные функции. Идея остаётся прежней, но вычисления становятся сложнее.

\subsection*{Дополнительные пояснения}

\textbf{В.: Почему этот метод называют методом конечных элементов?}

О.: Потому что каждый подотрезок рассматривается как "конечный элемент" с базисными функциями, определёнными локально.

\textbf{В.: Как на практике вычисляются интегралы $A_{jk}$ и $b_j$?}

О.: Для полиномиальных коэффициентов часто вычисляют точно аналитически. В общем случае используют численное интегрирование (формулы Гаусса и т.д.).

\textbf{В.: Какова сложность решения получившейся системы?}

О.: $O(N)$ операций с методом прогонки для трёхдиагональной матрицы. Без использования специальной структуры матрицы потребовалось бы $O(N^3)$ операций.

\textbf{В.: Как выбрать размер сетки?}

О.: В адаптивных методах сетка уплотняется в областях большой ошибки. Для простых задач достаточно равномерной сетки и апостериорной оценки погрешности.
